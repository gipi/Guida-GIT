\capitolo Linkografia

Questo strumento \`e in sviluppo vertiginoso e non esistono libri specifici
(tranne alcune eccezzioni) sull'argomento che \`e sempre in movimento; essendo
inoltre uno strumento nato e cresciuto in ambito ipertestuale, sembra pi\`u
consono creare una linkografia piuttosto che una bibliografia.
\bigskip
%\doublecolumns
\link http://git.or.cz/: Homepage di git, contenente gli archivi con le varie
versioni scaricabili, la documentazione; sito molto tecnico.

\link http://book.git-scm.com/: Pagina indirizzata alla creazione di un manuale
pi\`u user friendly rispetto alla documentazione dell'homepage.

\link http://members.cox.net/junkio/git/MaintNotes.html: Pagina di Junio Hamano che tratta della gestione del progetto e delle risorse relative.

\link http://whygitisbetterthanx.com/: Pagina di confronto fra gli altri
strumenti di versioning e git.

\link
http://betterexplained.com/articles/intro-to-distributed-version-control-illustrated/:
Guida visuale agli strumenti di revisione distribuiti.

\link http://tom.preston-werner.com/2009/05/19/the-git-parable.html:
``parabola'' sui principi che governano il design di git simulando la crezione di uno
strumento di versioning.

\link http://mendicantbug.com/2008/11/30/10-reasons-to-use-git-for-research/:
Una dissertazione dell'utilizzo di git in ambiente di ricerca.

\link http://vafer.org/blog/20080115011320:
Istruzioni abbastanza dettagliate per pubblicare un repository git attraverso
git daemon.

\link http://lwn.net/Articles/210045/: Articolo che descrive alcune
caratteristiche e funzionamento di git.

\link http://gitready.com/: Sito che raccoglie ``trucchi del giorno''
relativi all'uso di git.

\link
http://scie.nti.st/2007/11/14/hosting-git-repositories-the-easy-and-secure-way: Pagina con le istruzioni di installazione di \code{gitosis}.

\link http://skwpspace.com/git-workflows-book/: ``extremely unfinished beta book
in progress'' by Yan Pritzker.

\link http://toroid.org/ams/git-website-howto: Esempio di come utilizzare un
hook per fare il deploying di un sito web attraverso git.

\link http://www-cs-students.stanford.edu/~blynn/gitmagic/: Guida per l'utilizzo
di \code{GIT}.

\link http://gitref.org/: It's meant to be a quick reference for learning and remembering the most important and commonly used Git commands.

\vfill\eject
