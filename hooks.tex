\capitolo Hooks

Alcune volte \`e necessario customizzare i comportamenti di git in conseguenza o
in previsione di certe azioni.

\sezione applypatch-msg

Questo script viene eseguito nel momenti di applicare la

\sezione pre-applypatch

questo

\sezione post-applypatch

quello

\sezione pre-commit

ajsjs

\sezione prepare-commit-message

sss

\sezione commit-msg

djsosos

\sezione post-commit

dddd

\sezione pre-rebase

jdjdj

\sezione post-checkout

osajas

\sezione post-merge

djdjd

\sezione pre-receive

oaaiai

\sezione update

Invocato da \code{git-receive-pack} appena prima di aggiornare le referenze;
questo hook \`e eseguito una volta per ogni referenza secondo il seguente schema
\iniziacode
update refname oldobject newobject
|finecode
un valore di uscita zero di questo script permette l'aggiornamento della
referenza

\sezione post-receive

Questo hook \`e invocato da \code{git-receive-pack} nel repository remoto
(cio\`e dopo un \code{git push}) una volta che tutte le referenze sono state
aggiornate.

Per esempio pu\`o essere indicato per fare il deploying di un sito web a partire
dal suo repository bare, nel momento stesso in cui si pusha nel branch master
\iniziacode
#!/bin/sh

git checkout -f
|finecode

\sezione post-update

jdjdjd

\sezione pre-auto-gc

jjdjd
