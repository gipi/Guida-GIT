\figuranofloattikz[background rectangle/.style=
	{draw=blue!50,fill=blue!20,rounded corners=1ex},
  tag/.style={anchor=tip,single arrow, scale=.5,fill=yellow!50,rotate=315,draw},
	show background rectangle]
	\tikzstyle{every node}=[draw,fill=green,shape=circle,distance=5mm,scale=.6];
	\path (0:0cm)    node (v0);
	\node (v1) [right=of v0];
	\node (v2) [right=of v1];
	\node (v3) [right=of v2];
	\node (v4) [right=of v3];
	\node (v5) [right=of v4];
	\node (w1) [below right=of v5];
	\node (w2) [right=of w1];
	\node (z1) [below right=of v2];
	\node (z2) [right=of z1];
	\node (z3) [right=of z2];
	\draw [->] (v0) -- (v1);
	\draw [->] (v1) -- (v2);
	\draw [->] (v2) -- (v3);
	\draw [->] (v3) -- (v4);
	\draw [->] (v4) -- (v5);
	\draw [->] (v5) -- (w1);
	\draw [->] (w1) -- (w2);
	\draw [->] (v2) -- (z1);
	\draw [->] (z1) -- (z2);
	\draw [->] (z2) -- (z3);
% grazie ad "anchor=" posso impostare che la punta della freccia sia
% il punto rispetto a cui la disegna
	\node (master) at (v5.north)
    [anchor=tip,fill=red!50,single arrow,rotate=315,draw] {master};
	\node (topic) at (w2.north)
    [anchor=tip,fill=red!50,single arrow,rotate=315,draw] {topic};
	\node (next) at (z3.north)
    [anchor=tip,fill=red!50,single arrow,rotate=315,draw] {next};
\finefiguranofloattikz
