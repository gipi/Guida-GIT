\figuranofloattikz[show background rectangle]
	\path (0:0cm)    node (v0) {D};
	\node (v1) [right=of v0] {E};
	\node (v2) [right=of v1] {F};
	\node (v3) [right=of v2] {G};
	\node (w1) [below right=of v1] {A};
	\node (w2) [right=of w1] {B};
	\node (w3) [right=of w2] {C};
	\draw [->] (v0) -- (v1);
	\draw [->] (v1) -- (v2);
	\draw [->] (v2) -- (v3);
	\draw [->] (v1) -- (w1);
	\draw [->] (w1) -- (w2);
	\draw [->] (w2) -- (w3);
% grazie ad "anchor=" posso impostare che la punta della freccia sia
% il punto rispetto a cui la disegna
	\node[head] (master) at (v3.north) {master};
	\node[head] (topic) at (w3.north) {topic};
\finefiguranofloattikz
