\capitolo Workflow

Solo con l'elenco dei comandi disponibili all'interno di un repository git \`e
molto difficile che una persona possa comprendere effettivamente come usare
questo stupendo tool e quindi di seguito ci saranno degli esempi operativi di
come generalmente uno sviluppatore effettivamente usa git. Siccome le parole
spesso non sono il mio forte user\`o fortemente dei grafici per aiutarmi
(mi sono molto ispirato alle slide preparate da Junio Hamano).

\sezione Nuovo progetto

Si presuppone che stiate iniziando un progetto vostro e che la directory di
lavoro non contenga un precedente repository di git; non \`e importante che i
file che inserirete nel vostro progetti siano o no gi\`a presenti.

Per inizializzare il repository si usa il comando \code{git init}
\iniziacode
$ git init
Initialized empty Git repository in /path/to/git-repo/.git/
|finecode
poi magari si impostano le variabili (se non si \`e gi\`a fatto globalmente)
relative ai propri dati
\iniziacode
$ git config user.name "Nome Cognome"
$ git config user.email "email@dominio.org"
|finecode
Di seguito si inizia a modificare/copiare/creare i file che dovranno fare parte
della prima revisione del vostro progetto; una volta che si ha una prima
versione a voi congeniale del progetto, si aggiungono i file all'indice tramite
\code{git add}.

\`E anche possibile indicare tramite l'opzione \code{--shared} che si vuole
condividere il repository localmente, cio\'e con chi pu\`o accedere localmente
allo stesso filesystem dove si trova questo repository; dalla documentazione
si evince che \`e possibile passare dei parametri a questa opzione:

\iniziaelenco
\li \code{umask} (o \code{false}): usa \code{umask} per i permessi (in pratica
quello che fa quando \code{--shared} non \`e specificato).
\li \code{group} (o \code{true}): il repository viene reso accessibile in
scrittura al gruppo (con conseguente \code{g+sx}).
\li \code{all} (o \code{world} o \code{everybody}): come \code{group} ma rende
il repository accessibile in lettura a tutti.
\li \code{0xxx}: viene usato il numero ottale per impostare un valore custom per
\code{umask}.
\fineelenco

\sezione Progetto gi\`a esistente

In questo caso si tratta di lavorare su un repository non sviluppato da voi
direttamente ma a cui probabilmente siete interessati a partecipare, magari con
qualche patch ben congegnata. Il primo passo da effettuare \`e la ``clonazione''
del repository altrui tramite il seguente comando
\iniziacode
$ git clone http://dominio/repo.git
|finecode
Questo crea una directory di nome \code{repo} nella directory dove \`e stato
lanciato il comando a meno dell'uso dell'opzione \code{--bare} che genera invece
una directory \code{repo.git} senza un working tree. Questo ultimo comando \`e
utile per generare repository da condividere (c'\`e anche l'opzione \code{--shared}).

A questo punto magari si \`e interessati a conoscere il numero di branch
presenti sul repo remoto: per esempio questo succede per il codice originale di
git
\iniziacode
$ git branch -r
  origin/HEAD
  origin/html
  origin/maint
  origin/man
  origin/master
  origin/next
  origin/pu
  origin/todo
|finecode
Metodo standard di lavoro su codice altrui \`e creare un ramo di sviluppo locale
che sia correlato con i cambiamenti che si vogliono sottoporre al codice: per
inizializzare un branch alla corrente \code{HEAD} si deve eseguire il seguente
comando
\iniziacode
$ git checkout -b topic
|finecode

\sezione Facciamo la storia

Il programmare e/o lavorare ad un progetto implica la scrittura e la modifica
effettiva di file o quant'altro che devono poi essere registrati nella storia
del progetto stesso; come detto nel primo capitolo, \git\ fa uso del cosidetto
indice che permette di ``prendere da parte'' dei cambiamenti per inserirli nella
storia. Man mano che i cambiamenti effettuati soddisfano i requisiti minimi per 
un cambiamento li si aggiunge all'indice in modo da aggiornare il tree contenuto
nell'indice.

Una volta che i cambiamenti sono idonei a rappresentare una unit\`a specifica di
modifica si possono \evidenzia<committare>: questo andr\`a a creare dei nuovi
oggetti nel database in base a quali blob e tree sono stati modificati rispetto
al commit genitore; infine \HEAD\ e il branch corrente vengono aggiornati in
maniera da puntare a questo nuovo commit.

\iniziacode
$ git add file.txt
$ git commit -m 'descrizione del commit'
|finecode

\sezione Merging

Ci sono delle occasioni in cui rebasare un branch rispetto ad un altro non \`e
possibile in quanto, per esempio, il repository \`e stato esposto al pubblico ed
\`e cattiva creanza cambiare la storia senza un buon motivo. D'altro canto ci
sono procedure per cui il merge \`e l'operazione standard
(\code{pull}$=$\code{fetch}$+$\code{merge}) e quindi conviene conoscere il modo
un cui gestirlo in maniera adeguata.

Il nostro SCM usa il cosidetto ``three-way'' merge come metodo di base, cio\`e
utilizza tre versioni per ogni file: le versioni di cui si sta cercando di
eseguire il merge ed un antenato comune come mostrato nel grafo seguente
\input merge-scheme
Noi desideriamo tutti i cambiamenti che ci sono fra
\tikz\node[scale=.4,draw,fill=green,shape=circle]{A}; e
\tikz\node[scale=.4,draw,fill=green,shape=circle]{O}; oltre che fra
\tikz\node[scale=.4,draw,fill=green,shape=circle]{B}; e
\tikz\node[scale=.4,draw,fill=green,shape=circle]{O};. Ovviamente non \`e sempre
cos\`i facile in quanto nel repo possono essere gi\`a presenti dei merge
incrociati (\evidenzia<criss-cross merge>) che non rendono unici l'antenato ed
in questo caso si effettua un merge ``ricorsivo'' in cui si trova un antenato
successivo dei precedenti.

Per trovare il risultato finale dall'antenato comune, si tiene in considerazione
\orderedlist
\li se solo un lato cambia, si prende quello
\li se entrambi cambiano e portano allo stesso risultato, si prende quello
\li altrimenti si ha un conflitto
\endorderedlist
quanto detto sopra si applica ai cambiamenti relativi al contenuto dei file ma
anche ai relativi path. La nuova configurazione andr\`a a creare un nuovo
commit avente due commit genitori come in questo schema
\input merge-scheme-final

A parte la scelta della strategia, alla fine possono esserci dei conflitti
durante l'operazione che devono essere risolti manualmente (Linus docet).

\iniziacode
$ git show :2:<conflicted file> # mostra il contenuto dal master
$ git show :3:<conflicted file> # mostra il contenuto dal locale

$ git checkout --ours <conflicted file>   # prende contenuto dal master
$ git checkout --theirs <conflicted file> # prende contenuto dal locale
|finecode

in versioni precedenti alla ? i comandi sopra possono essere scritti come

\iniziacode
$ git reset -- <conflicted file>
$ git checkout MERGE_HEAD <conflicted file> # prende contenuto dal locale
|finecode

\iniziacode
$ git reset -- <conflicted file>
$ git checkout ORIG_HEAD <conflicted file> # prende contenuto dal locale
|finecode

Esiste anche la possibilit\`a di usare l'opzione \code{--squash} per ottenere un
commit nel branch attuale con il contenuto del merge ma senza generare un vero e
proprio merge (il commit risultante ha solo un genitore).

\sezione Rebasing

Supponiamo di aver creato un branch locale di nome ``topic'' su cui effettuare i
nostri cambi e che nel frattempo il ramo upstream (di solito il ``master'') sia
cambiato e noi vogliamo aggiornare il ramo locale con questi, esiste una
possibilit\`a attraverso il cosidetto \evidenzia<rebasing>: la semantica del
programma \`e
\iniziacode
$ git rebase [--onto <newbase>] <upstream> [<branch>]
|finecode
Nel caso sia specificato \code{<branch>} avviene un checkout di questo, tutti i
cambiamenti che sono nel branch corrente ma non sono \code{<upstream>} vengono
salvati in un'area temporanea per poi essere riapplicati una volta fatto il
checkout di \code{<upstream>} (oppure di \code{<newbase>} se specificata
l'opzione \code{--onto}); ovviamente \`e possibile che il processo di merging
possa intopparsi, in tal caso si applica quanto detto nella sezione del merging
per la risoluzione dei problemi. Una volta che si \`e convinti del risultato si
usa \code{rebase --continue} oppure saltare il commit incriminato (\code{git
rebase --skip}) o addirittura riportare tutto come alla partenza (\code{git
rebase --abort}). In \code{ORIG\_HEAD} viene salvata la referenza alla
cima del branch prima del reset.

Facciamo qualche esempio pratico: supponiamo di
trovarci nella situazione illustrata sotto, con un ramo principale di sviluppo
\code{master} ed uno specifico di sviluppo locale (\code{topic})
\input rebase-start
ed effettuiamo una operazione del tipo
\iniziacode
$ git rebase master topic
|finecode
cos\`\i\ da ritrovarci in una situazione in cui il ramo \code{topic} ha tutte le
modifiche che erano presenti upstream (ovviamente essendo cambiato il contenuto,
cambier\`a lo sha1 che identifica i vari file che generavano tree, commit e via
dicendo). Questo pu\`o essere rappresentato graficamente con il seguente schema
\input rebase-simple
Il comando \code{rebase} ha molte opzioni che lo rendono versatile per poter
adattare rami di sviluppo tra loro;
interessante \`e l'uso dell'opzione \code{--onto}: supponiamo di trovarci in una
situazione in cui un determinato branch si trovi ad essere stato sviluppato a
partire da un altro branch precedente come dalla seguente figura
\input rebase-onto-inizio
eseguendo il comando
\iniziacode
$ git rebase --onto master next topic
|finecode
ci si ritrova in una situazione in cui solo il branch \code{topic} \`e stato
aggiornato lasciando \code{next} nella sua situazione precedente
\input rebase-onto-fine

\`E possibile inoltre raffinare l'azione di modifica di un branch tramite
l'opzione \code{--interactive}, \code{-i} di \code{git rebase}:

{\bf Quando si rebasa interattivamente passando per dei commit che
aggiungono dei file, ricordarsi di usare git status per evitare di perderli!!!}

\sezione Conflitti

Fin qui abbiamo solo accennato alla possibilit\`a che ci siano degli errori
durante i processi di rebasing e merging, tuttavia nella vita reale queste sono
le situazioni pi\`u problematiche da gestire, sopratutto per la possibilit\`a di
perdere tempo e lavoro nel ``fondere'' codice che magari non conoscete a fondo.

Quando avviene un conflitto succedono le seguenti cose
\iniziaelenco
\li\HEAD rimane immutata
\li I file che non hanno subito conflitti sono aggiornato sia nell'index che nel
working tree.
\li Nel caso in un file ci siano conflitti, l'index memorizza fino a tre
versioni di questo: \code{stage1} contiene la versione dall'antenato comune,
\code{stage2} dalla \HEAD, \code{stage3} dal branch remoto. Il working tree
contiene il risultato del 3-way merge con dei marker \code{<<<<}, \code{===} e
\code{>>>>}.
\fineelenco
A questo punto si possono fare due cose
\iniziaelenco
\li Si decide di non effettuare il merge: \code{git reset --hard HEAD}
riporter\`a l'index a combaciare con l'ultimo commit.
\li Risolvere i conflitti: si editano i file scegliendo a mano quali combiamenti
includere, usare \code{git add}, e poi committare il risultato.
\fineelenco

\includecode merge-conflict

che corrisponder\`a al seguente file

\includecode hello-conflict

\sezione Rerere

``reuse recorded resolution'', \code{rerere.enabled}

Partiamo da un caso semplice di esempio per capire come poterlo usare nei casi
reali: abbiamo due branch con un solo file nel repository; si cerca di fare
il merge di uno nell'altro ma ci sono dei conflitti che risolvo una volta ma
ho l'accortezza di registrarli tramite \code{rerere}. Una volta risolto e
committato resetto ad una situazione prima del merge, lo rieseguo ed uso
\code{rerere} per ottenere il WT con la soluzione precedente:

\lonelypage rerere

{\bf Esempio 2:} esempio di come utilizzare rerere per memorizzare un conflitto
risolto.

In caso il salvataggio di \code{rerere} non sia corretto per un merge
successivo, il buon Hamano nel seguente thread http://permalink.gmane.org/gmane.comp.version-control.git/192926
ha esplicitato il seguente schema

\iniziacode
$ git merge other-branch
... rerere kicks in; eyeball the results
... ah, my earlier resolution is no longer correct
$ edit $the_path
... test the result of manual edit in the context of the merged whole
... and be satisified
$ git rerere forget $the_path
$ git add $the_path
$ git commit
... rerere records the updated resolution
|finecode

\sezione Terzo collaboratore

Succede che ci sia qualcun'altro che lavora ad uno stesso repo ma che ha
sviluppato delle modifiche indipendenti e, per un motivo o per l'altro, non sono
state inserite nel repo principale;
\iniziacode
$ git remote add bob git://dominio.di.bob.org/path/to/repo.git
$ git fetch bob
$ git branch -r
$ git log -p ..FETCH_HEAD
|finecode
In seguito al \code{fetch}, nel vostro database saranno presenti gli oggetti del
repo di bob che non sono presenti nel vostro, insieme alle sue referenze
(visualizzabili con \code{gitk}); nel caso voi siate interessati ad un suo
branch potete utilizzare l'opzione \code{--track} del comando \code{checkout}
che vi permetter\`a di aggiornare con un pull quella referenza remote: in questa
particolare casistica i seguenti comandi sono equivalenti
\iniziacode
$ git checkout --track -b i-love-programming bob/i-love-programming
$ git checkout --track bob/i-love-programming
|finecode

\sezione Coito interrotto

Il lavorare su rami di sviluppo \`e molto interessante, ma siccome ci sono pi\`u
branch ma un solo working tree, spesso ci si ritrova a dover passare velocemente
da un ramo all'altro (magari per fissare un baco di sicurezza appena scoperto)
ma aver il WT sporco; a questo punto ci si ritrova con due possibilit\`a: o si
fa un commit da amendare in seguito oppure si usa \code{git stash}.

Uno stash tecnicamente \`e un commit il cui tree registra lo stato della working
directory; il suo primo genitore \`e il commit dell'\url{HEAD} al momento della
sua creazione, il secondo registra invece l'indice al momento della creazione
(cio\`e in pratica \`e un \code{merge commit}):
nella figura seguente \`e mostrato visivamente quanto affermato
\input stash-tree
dove
\tikz\node[scale=.4,draw,fill=green,shape=circle]{W};
rappresenta un commit con il contenuto del WT, mentre
\tikz\node[scale=.4,draw,fill=green,shape=circle]{I}; in maniera analoga
contiene quello dell'indice.


Di default il comando crea uno stash a partire da \url{HEAD}, tuttavia questo
comando a sua volta accetta vari sottocomandi

Da notare che nel caso noi effettuassimo delle modifiche ad un branch
sbagliato, sarebbe molto complicato passarle in quello giusto, ma come abbiamo
visto con lo stash questo \`e immediato.

\elemento\code{list}: Elenca gli stashes che si trovano nella coda (\`e
implementato tramite \code{git log -g refs/stash}).

\elemento\code{show}: Mostra i cambiamenti registrati in uno stash.

\elemento\code{save}: Crea un nuovo stash a partire dalla situazione attuale di
WT e index e li pulisce per farli coincidere con HEAD a meno che l'opzione \code{--index} sia usata ed in tal caso lascia invariato lo stato dell'index.
Aggiunta recentemente l'opzione \code{--patch} che permette di creare stashes
di porzioni di codice.

\elemento\code{branch}:

\elemento\code{pop}: Rimuove un singolo stash
(se non indicato come argomento usa l'ultimo) e lo applica eseguendo
l'operazione inversa di \code{save}; il WT deve combaciare con l'index.
Nel caso si usi l'opzione \code{--index} cerca di impostare anche i cambi dell'indice.

\elemento\code{apply}: Come \code{pop} ma non
rimuove lo stash dalla lista.

\elemento\code{drop}:

\elemento\code{clear}:

\elemento\code{create}: Genera uno stash e restituisce il suo sha1 senza
tuttavia referenziarlo nel ref namespace.

Un'altra possibilit\`a \`e il trovarsi a controllare che un build di una
applicazione avvenga in maniera pulita, avendo gi\`a inserito nell'indice i
cambiamenti per il prossimo commit: \code{git stash save --keep-index} salva
come descritto precedentemente, ma lasciando per l'appunto l'indice intatto. A
questo punto per esempio un \code{make} e qualche test sull'applicazione potr\`a
farci sincerare se effettivamente abbiamo messo tutto il necessario per il
prossimo commit. Poi in realt\`a ci sarebbero anche i file ``untracked'' che
andrebbero eliminati per evitare falsi positivi (tipo file che dovrebbero essere
nell'indice ma non lo sono); \code{stash} pu\`o aiutare anche per questo: dopo
avere stashato i contenuti tracked, aggiungete all'indice quelli untracked e
stashate quelli.

\sezione Scova il baco

In alcuni casi ci ritroviamo con del codice avente dei problemi di vario tipo e
per un qualche motivo (magari proprio perch\'e il codice \`e nostro, oppure
perch\'e quel baco ci sta fottendo il filesystem) siamo noi a dovercene occupare
e anche in questo git ci d\`a una mano non indifferente.

\input bisect-scheme.tex

Questo comando \`e molto versatile, anche perch\`e possiede un certo numero di
sottocomandi che permettono una scelta mirata delle operazioni da compiere:

\elemento\code{visualize}: Attraverso \code{gitk} mostra le revisioni in ballo

\elemento\code{log}: 

\elemento\code{replay}: 

\elemento\code{run}: Usa uno script che testa per ogni revisione la sua
idoneit\`a; lo script usa il suo exit status per indicare a \code{git bisect} se
la revisione in esame \`e corretta o no.

\elemento\code{skip}: Salta una revisione.

\sezione Riscrivere la storia

Possono esserci vari casi in cui ci si ritrova a dover tornare sui nostri passi,
vuoi per un messaggio di commit sbagliato, per aver dimenticato un file oppure
ci si ritrova con un progetto, sviluppato magari internamente ad
una azienda, in cui sono presenti alcuni file che non sono più necessari e che
devono rimanere sconosciuti a persone esterne; si ha quindi la necessit\`a di
doter riscrivere la storia, magari totalmente.

Nel primo caso ci si aiuta con un mitico \code{git commit --amend} che riscrive
l'ultimo commit con il contenuto attuale dell'indice.

Nel secondo caso, molto pi\`u ``pericoloso'' in quanto deve modificare le
relazione fra blob/tree/commit esistenti, creandone anche di nuovi, non
modificando le parti utili; a questo \`e utile \code{git filter-branch}:
per modificare il nome e la mail dell'autore in tutti i commit
\iniziacode
$ git filter-branch --env-filter \
  "export GIT_AUTHOR_NAME='Nome Cognome' \
  GIT_AUTHOR_EMAIL='email@dominio.org' \
  GIT_COMMITTER_NAME='Nome Cognome' \
  GIT_COMMITTER_EMAIL=email@dominio.org"
|finecode
oppure per eliminare un file che non si desidera presentare nel repository
\iniziacode
git filter-branch --index-filter 'git rm --cached <file>' HEAD
|finecode

Caso particolare da tenere in considerazione \`e quello in cui si vuole
agire sul commit iniziale (quello che non ha genitori).

\url{http://stackoverflow.com/questions/2119480/changing-the-message-of-the-first-commit-git}

\sezione Quando qualcosa va storto

Pu\`o succedere che mentre si compiono azioni che vanno a riscrivere un branch
tramite rebasing o merging, il risultato non \`e quello che ci si aspettava,
vuoi per un errore nell'esecuzione dei comandi. In questo caso si pu\`o
utilizzare il comando \code{reflog} che tiene il log dei movimenti delle
referenze; di default il comando mostra il movimento di \code{HEAD} (subito
sotto si mostra l'output di \code{reflog} per il codice di questa guida)

\includecode reflog

Il fatto fondamentale \`e che ogni cambiamento che noi apportiamo alla struttura
DAG pu\`o essere recuperata in quanto ogni oggetto viene mantenuto nel database
fino a che comandi di pulizia (come per esempio \code{gc}) non vengono eseguiti.
La prima colonna indica lo \code{sha1} del commit in cui si trova a quel punto
\code{HEAD}, la seconda invece indica in quale ordine fosse; di seguito viene
indicato in quale circostanza \`e avvenuto il cambiamento (commit, rebasing,
merging, amending). Una volta trovata la referenza che recupera quanto perso, si
pu\`o eseguire un \code{git checkout -b NOMEBRANCH REF} per avere un branch con
le caratteristiche utili.

Ricordarsi che la prima colonna indica lo sha1 del risultato del processo.
