\capitolo Coesistenza

\sezione SVN

\`E anche possibile importare da un progetto \code{SVN}

\iniziacode
$ git svn clone --stdlayout http://path/to/svn/project/root
|finecode
questo dovrebbe importare \code{trunk} e i vari \code{branches}/\code{tags}. Se
il layout non ha un layout standard \`e possibile indicare come opzione di
\code{init} una o pi\`u delle opzioni \code{--branches}, \code{--tags}.

\sezione Mercurial

\code{Mercurial} \`e un altro sistema di versioning distribuito, molto
usato.

\`E possibile con

\iniziacode
$ git clone git://repo.or.cz/fast-export.git
$ cd fast-export
$ ./hg-fast-export.sh
Usage: hg fast-export.sh [--quiet] [-r <repo>] [--force] [-m <max>] [-s] [-A
<file>] [-M <name>] [-o <name>]

Import hg repository <repo> up to either tip or <max>
If <repo> is omitted, use last hg repository as obtained from state file,
GIT_DIR/hg2git-state by default.

Note: The argument order matters.

Options:
    -m      Maximum revision to import
    --quiet Passed to git-fast-import(1)
    -s      Enable parsing Signed-off-by lines
    -A      Read author map from file
            (Same as in git-svnimport(1) and git-cvsimport(1))
    -r      Mercurial repository to import
    -M      Set the default branch name (default to 'master')
    -o      Use <name> as branch namespace to track upstream (eg 'origin')
    --force Ignore validation errors when converting, and pass --force
            to git-fast-import(1)
|finecode

Se abbiamo un repository in \url{/opt/parcel} che \`e un progetto mantenuto
con mercurial dobbiamo creare un repository git vuoto e lanciare da quella
directory \code{hg-fast-export.sh}

\iniziacode
$ mkdir parcel-git && cd parcel-git
$ git init
$ /path/to/hg-fast-export.sh -r /path/to/hg/repo
|finecode
