\capitolo Sharing

Caratteristica indispensabile per un programma di revisione distribuita, \`e
quella di poter condividere semplicemente il proprio lavoro con gli altri, in
maniera indipendente ed autonoma.

\sezione Installazione servizio tramite inetd

Prima di tutto si controlla che ci sia la seguente riga in \code{/etc/services}
\iniziacode
git 9418/tcp
|finecode
che sta ad indicare che il servizio \code{git://} viene offerto alla porta 9418
tramite protocollo \code{TCP}; il modo pi\`u veloce di tirare su il servizio \`e
tramite il ``super'' demone \code{inetd(8)} che si preoccupa di gestire per noi
il programma \code{git daemon}. Nel file \code{/etc/inetd.conf} inserire quanto
indicato sotto (in un'unica riga):
\iniziacode
git stream  tcp nowait  nobody  /usr/bin/git
  git daemon --inetd --verbose --export-all --interpolated-path=/var/www/git/%D
  /var/www/git/project1
  /var/www/project2
|finecode
Di seguito si riavvia \code{inetd} lanciando un bel \code{SIGHUP} tramite
\code{pkill -HUP inetd} e per i test successivi si controlla il file
\code{/var/log/syslog}. Per motivi di sicurezza sarebbe utile che il repo venga
clonato tramite utente \code{nobody}. Eventuali problemi controllateli con
\url{/var/log/syslog}

Per utilizzare la filosofia secondo la quale \`e meglio dotare di meno permessi
possibili un programma che gira come demone, \`e preferibile creare un utente di
sistema di nome \url{git} che abbia una shell la quale permetta solo di pushare
e pullare (\url{git-shell}) e senza password (inserire nel file
\url{/home/git/.ssh/authorized_keys} le chiavi pubbliche a cui \`e permesso di
accedere in push).
\iniziacode
adduser --system --shell /usr/bin/git-shell --gecos 'git version control'
--group --disabled-password --home /home/git git
|finecode

\sezione Installazione servizio tramite runlevel

Al posto di usare \url{inetd} si pu\`o creare un servizio a parte utilizzando
una cosa similare al seguente script il quale utilizza magistralmente le regole
standard che potete trovare in \url{/lib/lsb/init-functions} ed il mitico
comando \url{start-stop-daemon(8)}

\iniziacode
#!/bin/sh

test -f /usr/bin/git-daemon || exit 0

. /lib/lsb/init-functions

GITDAEMON_OPTIONS="--reuseaddr --verbose --base-path=/home/git/repositories/ --detach"

case "$1" in
start)  log_daemon_msg "Starting git-daemon"

        start-stop-daemon --start -c git:git --quiet --background \
                     --exec /usr/bin/git-daemon -- ${GITDAEMON_OPTIONS}

        log_end_msg $?
        ;;
stop)   log_daemon_msg "Stopping git-daemon"

        start-stop-daemon --stop --quiet --name git-daemon

        log_end_msg $?
        ;;
*)      log_action_msg "Usage: /etc/init.d/git-daemon {start||stop}"
        exit 2
        ;;
esac
exit 0
|finecode
Salvato come \url{/etc/init.d/git-daemon} eseguiamo le seguenti operazioni
\iniziacode
# chmod +x
# update-rc.d git-daemon defaults
|finecode
per renderlo eseguibile ed impostare l'avvio nel runlevel standard.

Il seguente script serve per inizializzare un repository vuoto su cui pushare
successivamente (un po' come succede con \url{gitorious}); \`e inteso da
utilizzare tramite \url{root} (purtroppo se l'utente \url{git} non ha molti
poteri vista la shell limitata di cui \`e dotato, tocca al superutente fare
tutto). Per pushare usare \url{git push git@}
\iniziacode
#!/bin/sh

REPOS_PATH=/home/git/repositories/

function log_err(){
  echo $1
  exit 1
}

function create_dir(){
  echo "+ creating dir '$1'"
  mkdir $1 || log_err "I couldn't create $1"
}

function init_git_dir(){
  echo "+ create empty git repo"
  GIT_DIR=$1 git-init-db || log_err "git init failed"
}

function change_ownership(){
  echo "+ change ownership"
  chown -R git:git $1 || log_err "chown failed"
}

function enable_export(){
  echo "+ enabling export"
  touch $1/git-daemon-export-ok
}

if [ -z "$1" ]
then
  echo "oh my god! no args?"
  exit 1
fi

NEW_REPO=$1
NEW_REPO_PATH=${REPOS_PATH}/${NEW_REPO}

create_dir ${NEW_REPO_PATH}
init_git_dir ${NEW_REPO_PATH}
enable_export ${NEW_REPO_PATH}
change_ownership ${NEW_REPO_PATH}
|finecode

\sezione Mailing list

Il progetto stesso di git \`e sviluppato da una comunit\`a open source, da
persone localizzate geograficamente nei luoghi pi\`u disparati, basti pensare
che il maintainer \`e giapponese, mentre i collaboratori sono prevalentemente
europei e americani. Il confronto sul codice, le decisioni riguardanti le
modifiche avvengono via mailing list (che personalmente consiglio di seguire per
avere idea di cosa voglia dire collaborare in un progetto di questo tipo)
attraverso gli strumenti che questo programma stesso mette a disposizione.

Il comando effettivo \`e \code{send-email} che pu\`o appoggiarsi ad un programma
alla \code{sendmail} per produrre il flusso di mail voluto; nel caso non abbiate
il programma configurato \`e possibile usare \code{msmtp} con il seguente file
di configurazione

\includecode msmtprc

impostando come valore in \code{sendemail.server}, oppure
\code{--smtp-server}, il path alla sua installazione.

Un'altra possibilit\`a \`e usare il comando \code{imap-send} che invia patches
passate tramite stdin ad una cartella \code{IMAP}

\iniziacode
[imap]
        folder = "INBOX.Drafts"
        host = imaps://awesome.dominio.org
        user = mario.rossi@awesome.dominio.org
|finecode

Da parte di chi riceve le patch \'e possibile utilizzare invece il comando
\code{git am} per applicarle al proprio repository.

\sezione Gitolite

\code{Gitolite} \`e un layer di amministrazione di accesso ai repository tramite
un unico account accessibile da \code{ssh}; il repository del codice sorgente
si trova su \url{http://github.com} al seguente indirizzo.
\iniziacode
$ git clone git://github.com/sitaramc/gitolite.git
|finecode
Ancora su \url{http://github.com} si trova la documentazione, pi\`u precisamente
all'indirizzo
\medskip
\centerline{\url{http://sitaramc.github.com/gitolite/master-toc.html}}
\medskip
Per l'installazione \`e consigliato creare un utente apposito nel sistema con
meno privilegi possibili. Per eseguire l'installazione vera e propria
eseguire il comando
\iniziacode
# adduser gitolite --disabled-password --home /var/gitolite
# su - gitolite
$ git clone git://github.com/sitaramc/gitolite.git
$ mkdir bin
$ gitolite/install -to ~/bin
$ gitolite setup -pk YourName.pub # YourName.pub è la propria chiave pubblica
|finecode
A questo punto \`e possibile nella macchina che dovr\`a gestire i repository
clonare il repository \code{gitolite-admin} che \`e stato appena creato con i
diversi file di configurazione:
\iniziacode
$ git clone gitolite@xxx:gitolite-admin.git
|finecode
vi creer\`a il seguente repo
\iniziacode
$ tree gitolite-admin/
gitolite-admin/
||-- conf
||   `-- gitolite.conf
`-- keydir
    `-- YourName.pub
|finecode
In alcuni casi ci possono essere problemi con il client \code{ssh} che non trova la chiave giusta: \`e
consigliabile inserire delle righe come le seguenti nel file \code{.ssh/config}
\iniziacode
Host xxx
  User gitolite
  HostName 192.168.0.5
  IdentityFile /home/gipi/.ssh/id_rsa
  IdentitiesOnly yes
|finecode




\sezione Bundle

Nel caso non si abbia la disponibilit\`a di un servizio e/o il collegamento
tramite rete \`e possibile effettuare le operazioni tramite il comando
\code{bundle}.

\iniziacode
user@remote: $ git bundle create repo.bundle HEAD
... sposta repo.bundle in local
user@local: $ git clone repo.bundle
|finecode
% http://stackoverflow.com/questions/3635952/how-to-use-git-bundle-for-keeping-development-in-sync
\iniziacode
# on hostA, the initial home of the repo
hostA$ git bundle create hostA.bundle --branches --tags

# transfer the bundle to hostB, and continue:
hostB$ git clone /path/to/hostA.bundle my-repo
# you now have a clone, complete with remote branches and tags
# just to make it a little more obvious, rename the remote:
hostB$ git remote rename origin hostA

# make some commits on hostB; time to transfer back to hostA
# use the known master branch of hostA as a basis
hostB$ git bundle create hostB.bundle ^hostA/master --branches --tags

# copy the bundle back over to hostA and continue:
hostA$ git remote add hostB /path/to/hostB.bundle
# fetch all the refs from the remote (creating remote branches like hostB/master)
hostA$ git fetch hostB
# pull from hostB's master, for example
hostA$ git pull

# make some commits on hostA; time to transfer to hostB
# again, use the known master branch as a basis
hostA$ git bundle create hostA.bundle ^hostB/master --branches --tags
# copy the bundle to hostB, **replacing** the original bundle
# update all the refs
hostB$ git fetch hostA

# and so on and so on
|finecode
