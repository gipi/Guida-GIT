\capitolo Scatole cinesi

Un problema che non era stato affrontato dal principio dello sviluppo di \git
era la possibilit\`a di poter tenere sotto versioning un progetto composto da
altri sottoprogetti mantenuti tramite \git stesso. In seguito \`e stata
sviluppata un comando che permette questo.

\sezione Submodule

Un discorso a parte meritano i submodule, cio\`e il modo che ha git di gestire
un insieme di repository in uno principale.

Per facilitare prove si pu\`o usare un repository con molti submodules, quale
per esempio quello di \code{xorg}
\iniziacode
$ git clone git://people.freedesktop.org/~whot/xorg.git
$ cd xorg
$ git submodule update --init
|finecode

\sezione Subtree

Esiste una particolare strategia da usare per il merge che permette di tenere
distinti dei progetti per poi unirli come ``subdirectories''. Supponiamo di
avere un progetto raggiungibile ad un \URL \verbatim <repo>|endverbatim e che
desideriamo inserire il contenuto del branch \verbatim <branchname>|endverbatim
di quel repo (in maniera da tenere anche il versioning of course) nella
directory \url{another-project/}

\iniziacode
$ git remote add -f <remotename> <repo>
$ git merge -s ours --no-commit <remotename>/<branchname>
$ git read-tree --prefix=another-project/ -u <remotename>/<branchname>
$ git commit
|finecode
In seguito si potranno scaricare i nuovi commit usando la strategia
\code{subtree} con il comando \code{pull}

\iniziacode
$ git pull -s subtree <remotename> master
|finecode
\`E stato anche presentato un nuovo comando \git che permetta di rendere questi
passaggi pi\`u semplici; inoltre \`e possibile ri-dividere il repo nei suoi
progetti che lo compongono. Ne parlano qui
\url{http://alumnit.ca/~apenwarr/log/?m=200904#30}.
