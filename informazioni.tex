\capitolo Ottenere informazioni

In maniera equivalente alla gestione di modifiche del codice, \`e importante
poter recuperare informazioni memorizzate direttamente o indirettamente
all'interno del codice, come per esempio autori, struttura dell'albero dei
sorgenti, quali file sono stati modificati e quando etc...

All'interno di \code{git} esistono famiglie di comandi che sono stati pensati a
questo scopo e molto malleabili da questo punto di vista.

\sezione Log

Il pi\`u alla mano \`e sicuramente \code{git log} che visualizza a partire da
una data referenza (se non la si indica parte da \code{HEAD}) la struttura di
grafo fra commit. Per esempio se si \`e pullato dal master di un repository i
seguenti comandi danno diverse informazioni
\iniziacode
$ git log master..origin/master
$ git log origin/master..master
$ git log origin/master...master
|finecode
nel primo caso vengono visualizzati i cambiamenti fatti dagli altri, nel secondo
quello che si ha in locale senza mostrare i cambiamenti degli altri. L'ultimo
mostra entrambi.

\figuratikz[show background rectangle]
	\path (0:0cm)    node (v0) {};
	\node[commit] (v1) [right=of v0]
{
\tikzcommit{Cambiamento epocale nella programmazione}
};
	\node (v2) [right=of v1];
\endfiguratikz[log]

\iniziacode
$ git log --pretty=oneline --stat \| \\
   sed -n 's/^ \([0-9]*\) files changed, \([0-9]\) insertions(+), \([0-9]\) deletions(-)/\1 \2 \3/p'
|finecode

\sezione Differenze

Quello che effettivamente noi registriamo con il nostro repository \`e
l'evoluzione del codice e il messaggio nel commit rappresenta solo un riassunto
delle modifiche apportate al codice; per conoscere effettivamente come si \`e
modificato oppure si sta modificando il codice git ha la sua versione del tool
standard nel mondo unix per questo scopo: diff. Ecco un piccolo assaggio delle
sue funzionalit\`a:

\iniziacode
$ git diff           # differenze fra working tree e indice
$ git diff --cached  # differenze fra indice e ultimo commit
$ git diff HEAD      # differenze fra working tree e ultimo commit
|finecode

\sezione Autori

Ovviamente le informazioni immagazzinate insieme al commit sono molto utili in
certi ambiti, per esempio per scoprire chi originariamente ha apportato una
certa modifica ad una certa riga: a questo viene in aiuto \code{git blame}.

Una opzione comoda \`e senz'altro \code{-s} che permette di evitare di
visualizzare autori e timestamp (altrimenti non sta negli 80 caratteri
di un terminale).
