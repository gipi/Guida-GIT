\capitolo Comandi

Il modo in cui avviene l'interazione con un repository git \`e attraverso i
(sotto-)comandi propri di questo SCM e proprio per una scelta implementativa
questi sono principalmente divisi in due categorie principali: i cosidetti
\evidenzia<porcelain> e \evidenzia<plumbing>, i primi sono comandi intesi da
essere utilizzati normalmente, mentre i secondi da script o richiamati dai primi.

Qui di seguito diamo un elenco dei comandi divisi per le rispettive aree di
interesse: comandi che modificano il repository (lo inizializzano, lo
``puliscono'' etc..), comandi che modificano l'indice e quelli che danno
informazioni. Li elencher\`o omettendo il termine git.

\sezione Comandi per la gestione repository

\elemento\code{init}: Inizializza il repository creando la directory {\tt .git}
e tutte le relative sottodirectory.

\elemento\code{fetch}: Esegue il download di oggetti e referenze da un altro
repository.

\elemento\code{pull}: Come il precedente, ma esegue anche un merge con il ramo
di lavoro del repo locale.

\elemento \code{clone}: Inizializza un repository creando una copia di un altro
repository passato come argomento; il repository pu\`o essere sia locale che
remoto (nel primo caso il database degli oggetti pu\`o essere in comune).
Questa copia viene creata in una nuova directory e crea dei branch locali per
ogni branch remoto (visualizzabili tramite \code{git branch -r}).

\elemento \code{gc}: Comando che gestisce la pulizia e l'ordine del repository;
si occupa di eliminare i file non pi\`u referenziati, crea un database degli
oggetti immagazzinati attraverso la creazione dei file \code{.pack} che
contengono una rappresentazione degli oggetti attraversa dei delta. Notare che
finch\`e non viene eseguito questo comando esplicitamente, ogni oggetto che un
tempo si trovava nel db \`e ancora presente.

\sezione Comandi che modificano l'indice ed il working tree

Come \`e stato detto, l'indice \`e usato come luogo temporaneo per memorizzare i
cambiamenti che si intendono poi memorizzare nello stato del repository.

\elemento\code{add}: Aggiunge il file passato come argomento all'indice come
oggetto da essere memorizzato nel prossimo commit.

\elemento\code{reset}: Riporta \code{HEAD} ad uno stato specifico; pu\`o
eventualemente resettare l'indice e/o il working tree.

\elemento\code{clean}: Rimuove file non nell'indice dal working tree.

\elemento\code{stash}: Salva lo stato dell'indice in una referenza temporanea
nel caso si necessiti un ambiente pulito in poco tempo.

\elemento\code{checkout}: Imposta il working tree ad un determinato stato. \`E
lo strumento base per il passaaggio da un branch all'altro (quando gi\`a
esistenti).

\elemento\code{am}: Applica delle patch ottenute tramite email estraendole da
una mailbox ed \`e possibile eseguirlo interattivamente.

\elemento\code{cherry-pick}: Applica i cambiamenti introdotti da un commit
nell'attuale indice e working tree creando un nuovo commit; prevalentemente
usato per traslare da un branch locale ad un altro singole modifiche senza
effettuare rebasing o merging.

\sezione Comandi per condividere

\elemento\code{format-patch}: Genera una (serie di) patch formattate in
maniera da poter essere mandati tramite mail. Prende molti degli argomenti che
digerisce \code{diff}.

\elemento\code{send-mail}: Invia patch come email, molto comodo nel caso in cui
il workflow comprenda la discussione dello sviluppo software via mail.

\elemento\code{bundle}: Genera, verifica o descrive un archivio comprendente
oggetti e referenze di un repository.

\sezione Comandi che danno informazioni

\`E molto importante, anche ai fini della comprensione del codice e delle sue
modifiche, avere sott'occhio quale \`e la descrizione data dall'autore di una
modifica, le righe modificate o anche spostate da un file all'altro

\elemento\code{diff}: Visualizza differenze fra commit visualizzandoli nel
formato \evidenzia<unidiff>\notapiepagina{Lo ``unified format'' ha nella sua
intestazione una riga del tipo \code{@@ -R +R @@}, dove \code{R}
rappresenta un range attraverso la sequenza $l,s$ dove $l$ \`e la linea
di partenza
della modifica ed $s$ il numero di linee cambiate. Per ulteriori informazioni ci
si pu\`o riferire a \url{http://en.wikipedia.org/wiki/Diff#Unified_format}}
modificato.

\elemento\code{log}: Visualizza informazioni riguardanti i commit in senso
temporale.

\elemento\code{reflog}: Visualizza informazioni riguardo ai movimenti delle
``punte'' dei branch; \`e molto utile per recuperare cambiamenti non pi\`u
direttamente accessibili come per esempi quelli precedenti ad un rebasing.

\elemento\code{grep}: In maniera similare al tool da riga di comando, cerca
all'interno dei file del repo se esiste una data stringa.

\elemento\code{ls-files}: Restituisce informazioni riguardanti i file presenti
nell'indice e nel working tree.

\elemento\code{ls-tree}: Come il precedente, ma limitato ai tree; l'output
assomiglia molto a quello di \code{ls -l} nella normale shell \url{UNIX} like.

\elemento\code{blame}: Mostra dato un file, per ogni riga
a quale commit \`e riconducibile la stessa; molto utile per contattare lo
sviluppatore del codice che ci interessa.

\elemento\code{annotate}: Praticamente come il precedente ed esistente solo per
retrocompatibilit\`a, fornendo un nome similare a comandi presenti in altri SCM.

\sezione Comandi che configurano

\elemento\code{config}: \`E il programma principale attraverso cui impostare il
comportamento di \code{git} per i vostri scopi.

\sezione Cose che non sono comandi

\elemento\code{.gitignore}: File tramite cui istruire \code{git} che alcune
tipologie di file (specificabili anche tramite dei pattern) non sono da tenere
in considerazione; principalmente da usare per evitare file binari e
temporanei.

\elemento\code{.gitattributes}: File tramite cui istruire \code{git}

\elemento\code{hooks}: Script posizionati in \url{$GIT_DIR/hooks/} e che sono
preposti per specifiche azioni da compiersi in determinate situazioni.

