\capitolo Comandi

Qui di seguito diamo un elenco dei comandi divisi per le rispettive aree di
interesse: comandi che modificano il repository (lo inizializzano, lo
``puliscono'' etc..), comandi che modificano l'indice e quelli che danno
informazioni. Li elencher\`o omettendo il termine git.

\sezione Comandi per la gestione repository

\elemento \code{init}: Inizializza il repository creando la directory {\tt .git}
e tutte le relative sottodirectory.

\elemento \code{clone}: Inizializza un repository creando una copia di un altro
repository passato come argomento; il repository pu\`o essere sia locale che
remoto e la copia pu\`o avere il database degli oggetti in comune nel caso sia
locale.

\elemento \code{gc}: Comando che gestisce la pulizia e l'ordine del repository;
si occupa di eliminare i file non pi\`u referenziati, crea un database degli
oggetti immagazzinati attraverso la creazione dei file \code{.pack} che
contengono una rappresentazione degli oggetti attraversa dei delta.

\sezione Comandi che modificano l'indice

Come \`e stato detto, l'indice \`e usato come luogo temporaneo per memorizzare i
cambiamenti che si intendono poi memorizzare nello stato del repository.

\elemento\code{add}: Aggiunge il file passato come argomento all'indice come
oggetto da essere memorizzato nel prossimo commit.

\sezione Comandi che danno informazioni

\`E molto importante, anche ai fini della comprensione del codice e delle sue
modifiche, avere sott'occhio quale \`e la descrizione data dall'autore di una
modifica, le righe modificate o anche spostate da un file all'altro
