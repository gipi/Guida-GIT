\input macro
%\parfillskip 10pt plus 1fil
% capitoli
\line{\capitolofont GIT}
\vfil
\licenza
\eject
\input getting-started
\input comandi
\input workflow
\input sharing

\input submodule

\capitolo Miscellanea

\sezione Debugging del push

Per qualche oscura ragione potrebbe non funzionare correttamente il push tramite
\code{ssh} e quindi di seguito un metodo molto interessante
\iniziacode
$ echo "ssh -v $@" > ssh_debug_wrapper
$ GIT_SSH="./ssh_debug_wrapper" git push
|endverbatim
\finecode

\sezione Creare branch indipendente o commit senza parenti

Per creare un commit chiamato \code{mybranch} senza parenti
(in pratica un branch slegato)
\iniziacode
$ git symbolic-ref HEAD refs/heads/mybranch
$ git rm --cached -r .
$ rm *
$ git commit --allow-empty
|endverbatim
\finecode

\sezione Low level

Per la compressione a basso livello viene usata la libreria zlib
(\url{http://www.zlib.net}) unica vera dipendenza di git; i loose object nel
database vengono compressi tramite esso ed \`e possibile ottenere un programma
per la decompressione a riga di comando alla pagina
\url{http://www.zlib.net/zpipe.c}. Una volta compilato \`e possibile ottenere la
decompressione di un oggetto tramite \code{zpipe -d < .git/objects/xx/} in
maniera analoga al comando \code{git cat-file -p SHA1} bench\'e il secondo dia
un risultato formattato in maniera molto pi\`u human-readable.

Volendo essere pi\`u ``creativi'' esiste il comando \code{hash-object} che crea
un blob ed il comando \code{mktree} che crea un \code{tree} salvandoli nella
directory \code{.git/objects/} e restituendo lo \code{sha1} relativo. Infine per
creare un commit esiste il comando \code{commit-tree} che prende come primo
argomento un tree (o pi\`u d'uno) e dallo \code{stdin} il messaggio relativo;
ovviamente ci viene restituito lo \code{sha1} del commit e salvato del database
l'oggetto relativo.

\capitolo Linkografia

Questo strumento \`e in sviluppo vertiginoso e non esistono libri specifici
(tranne alcune eccezzioni) sull'argomento che \`e sempre in movimento; essendo
inoltre uno strumento nato e cresciuto in ambito ipertestuale, sembra pi\`u
consono creare una linkografia piuttosto che una bibliografia.
\bigskip
%\doublecolumns
\link http://git.or.cz/: Homepage di git, contenente gli archivi con le varie
versioni scaricabili, la documentazione; sito molto tecnico.

\link http://book.git-scm.com/: Pagina indirizzata alla creazione di un manuale
pi\`u user friendly rispetto alla documentazione dell'homepage.

\link http://whygitisbetterthanx.com/: Pagina di confronto fra gli altri
strumenti di versioning e git.

\link
http://betterexplained.com/articles/intro-to-distributed-version-control-illustrated/:
Guida visuale agli strumenti di revisione distribuiti.

\link http://mendicantbug.com/2008/11/30/10-reasons-to-use-git-for-research/:
Una dissertazione dell'utilizzo di git in ambiente di ricerca.

\link http://vafer.org/blog/20080115011320:
Istruzioni abbastanza dettagliate per pubblicare un repository git attraverso
git daemon.

\link http://lwn.net/Articles/210045/: Articolo che descrive alcune
caratteristiche e funzionamento di git.

\vfill\eject

\end
