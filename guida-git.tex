\input eplain
\font\capitolofont=cmssdc10 at 30pt
\font\sezionefont=cmssdc10 at 15pt
%	from tex-by-topic pg211
\def\goodbreak{\par\penalty-500 } % encourage page break
\def\capitolo#1\par{%
	\vfill\supereject
	\noindent\line{\capitolofont #1\hfil}%
	\vskip 2cm
}%
\def\sezione#1\par{%
	\bigskip
	\goodbreak
	\noindent\line{\sezionefont #1\hfil}%
	\smallskip\noindent
}%
\def\elemento#1: #2\par{% from 'A plain Tex primer'
	\smallskip
	\bgroup
	\parindent=4cm
	\narrower
	\noindent\llap{\hbox to 4cm{\hfil\bf #1\hfil}}#2\par%
	\egroup
	\smallskip
}%
\def\code #1{{\tt #1}}
\def\iniziacode{\smallskip\noindent\line{\hrulefill}\smallskip}
\def\finecode{\smallskip\noindent\line{\hrulefill}\smallskip\noindent}
\def\evidenzia<#1>{{\bf #1}}
%%%%%%%%%%%%%%%%%%%%%%%%%%%%%%%%%%%%%%%%%%%%%%%%%%%%%%%%%%%
\capitolo Getting started

In questi anni si stanno affermando un certo numero di strumenti di tipo
informatico per gestire le revisioni del codice sorgente in tutte le possibili
sfaccettature: changelog, autori, righe modificate etc...

\sezione Glossario

Spesse volte la potenza di uno strumento si scontra con un certo numero di
concetti che \`e necessario assimilare prima di imparare ad usare
effettivamente un dato strumento e, perch\'e no, modificarlo a proprio uso e
consumo. In questa sezione definiamo alcuni termini utili per le successive
dissertazioni (alcuni termini saranno mantenuti in inglese perch\'e molto pi\`u
evocativi per me).

\elemento Repository: Collezione formata da un database di oggetti e referenze
relative a queste; praticamente implementata tramite una serie di sottodirectory
la cui radice \`e %\verbatim .git/ |endverbatim.

\elemento Working tree: rappresenta l'albero dei sorgenti fisicamente presenti
nel repository.


\end
