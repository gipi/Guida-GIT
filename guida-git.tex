\input macro
\input tikz-macro
%\IT
\font\commitfn=cmtt10 at 5pt
%\parfillskip 10pt plus 1fil
\parindent=10pt
% capitoli
\line{\capitolofont GIT}
\vfil
\licenza
\eject
\input getting-started
\input comandi
\input workflow
\input informazioni
\input sharing
\input submodule
\input attributes
\input config
\input hooks
\input interfaccie-grafiche
\input low-level
\input the-git-directory

\capitolo Miscellanea

\sezione Installazione non standard on the remote side

Magari sulla macchina remota l'installazione \`e in posizione non banale quindi
per evitare scleri

\iniziacode
$ git [clone || pull] --upload-pack /path/to/git-upload-pack [URL]
|finecode

\sezione Debugging del push

Per qualche oscura ragione potrebbe non funzionare correttamente il push tramite
\code{ssh} e quindi di seguito un metodo molto interessante
\iniziacode
$ echo "ssh -v -v -v $@" > ssh_debug_wrapper
$ GIT_SSH="./ssh_debug_wrapper" git push
|finecode

\sezione Avere permessi per referenza

Nella sezione \url{contrib/hooks/update-paranoid} esiste uno script che permette
di avere un accesso controllato pi\`u finemente di quanto permetta
\code{gitosis}. Trascrivo qui il contenuto della parte di documentazione
contenuta nel file stesso

\iniziacode
Invoked as: update refname old-sha1 new-sha1

This script is run by git-receive-pack once for each ref that the
client is trying to modify.  If we exit with a non-zero exit value
then the update for that particular ref is denied, but updates for
other refs in the same run of receive-pack may still be allowed.

We are run after the objects have been uploaded, but before the
ref is actually modified.  We take advantage of that fact when we
look for "new" commits and tags (the new objects won't show up in
`rev-list --all`).

This script loads and parses the content of the config file
"users/$this_user.acl" from the $acl_branch commit of $acl_git ODB.
The acl file is a git-config style file, but uses a slightly more
restricted syntax as the Perl parser contained within this script
is not nearly as permissive as git-config.

Example:

  [user]
    committer = John Doe <john.doe@example.com>
    committer = John R. Doe <john.doe@example.com>

  [repository "acls"]
    allow = heads/master
    allow = CDUR for heads/jd/
    allow = C    for ^tags/v\\d+$

For all new commit or tag objects the committer (or tagger) line
within the object must exactly match one of the user.committer
values listed in the acl file ("HEAD:users/$this_user.acl").

For a branch to be modified an allow line within the matching
repository section must be matched for both the refname and the
opcode.

Repository sections are matched on the basename of the repository
(after removing the .git suffix).

The opcode abbrevations are:

  C: create new ref
  D: delete existing ref
  U: fast-forward existing ref (no commit loss)
  R: rewind/rebase existing ref (commit loss)

if no opcodes are listed before the "for" keyword then "U" (for
fast-forward update only) is assumed as this is the most common
usage.

Refnames are matched by always assuming a prefix of "refs/".
This hook forbids pushing or deleting anything not under "refs/".


Refnames that start with ^ are Perl regular expressions, and the ^
is kept as part of the regexp.  \\ is needed to get just one \, so
\\d expands to \d in Perl.  The 3rd allow line above is an example.

Refnames that don't start with ^ but that end with / are prefix
matches (2nd allow line above); all other refnames are strict
equality matches (1st allow line).

Anything pushed to "heads/" (ok, really "refs/heads/") must be
a commit.  Tags are not permitted here.

Anything pushed to "tags/" (err, really "refs/tags/") must be an
annotated tag.  Commits, blobs, trees, etc. are not permitted here.
Annotated tag signatures aren't checked, nor are they required.

The special subrepository of 'info/new-commit-check' can
be created and used to allow users to push new commits and
tags from another local repository to this one, even if they
aren't the committer/tagger of those objects.  In a nut shell
the info/new-commit-check directory is a Git repository whose
objects/info/alternates file lists this repository and all other
possible sources, and whose refs subdirectory contains symlinks
to this repository's refs subdirectory, and to all other possible
sources refs subdirectories.  Yes, this means that you cannot
use packed-refs in those repositories as they won't be resolved
correctly.
|finecode


\sezione Creare branch indipendente o commit senza parenti

Per creare un commit chiamato \code{mybranch} senza parenti
(in pratica un branch slegato)
\iniziacode
$ git symbolic-ref HEAD refs/heads/mybranch    # crea una nuova referenza
$ rm .git/index                                # cancella il prossimo commit
$ git clean -fdx                               # cancella il working tree (opzionale)
$ git add <qualcosa>                           # aggiunge elementi per il commit
$ git commit                                   # commit senza parenti
|finecode

\sezione Tenere la storia pulita secondo Linus

In questa email l'amato creatore di git rende disponibile al pubblico la sua
interpretazione sulla pulizia della storia\notapiepagina{\url{http://www.mail-archive.com/dri-devel@lists.sourceforge.net/msg39091.html}}:

\iniziacode
Re: [git pull] drm-next

Linus Torvalds
Sun, 29 Mar 2009 14:48:18 -0700

On Sun, 29 Mar 2009, Dave Airlie wrote:
>
> My plans from now on are just to send you non-linear trees, whenever I 
> merge a patch into my next tree thats when it stays in there, I'll pull 
> Eric's tree directly into my tree and then I'll send the results, I 
> thought we cared about a clean merge history but as I said without some 
> document in the kernel tree I've up until now had no real idea what you 
> wanted.

I want clean history, but that really means (a) clean and (b) history.

People can (and probably should) rebase their _private_ trees (their own 
work). That's a _cleanup_. But never other peoples code. That's a "destroy 
history"

So the history part is fairly easy. There's only one major rule, and one 
minor clarification:

 - You must never EVER destroy other peoples history. You must not rebase 
   commits other people did. Basically, if it doesn't have your sign-off 
   on it, it's off limits: you can't rebase it, because it's not yours.

   Notice that this really is about other peoples _history_, not about 
   other peoples _code_. If they sent stuff to you as an emailed patch, 
   and you applied it with "git am -s", then it's their code, but it's 
   _your_ history.

   So you can go wild on the "git rebase" thing on it, even though you 
   didn't write the code, as long as the commit itself is your private 
   one.

 - Minor clarification to the rule: once you've published your history in 
   some public site, other people may be using it, and so now it's clearly 
   not your _private_ history any more.

   So the minor clarification really is that it's not just about "your 
   commit", it's also about it being private to your tree, and you haven't 
   pushed it out and announced it yet.

That's fairly straightforward, no?

Now the "clean" part is a bit more subtle, although the first rules are 
pretty obvious and easy:

 - Keep your own history readable

   Some people do this by just working things out in their head first, and 
   not making mistakes. but that's very rare, and for the rest of us, we 
   use "git rebase" etc while we work on our problems. 

   So "git rebase" is not wrong. But it's right only if it's YOUR VERY OWN 
   PRIVATE git tree.

 - Don't expose your crap.

   This means: if you're still in the "git rebase" phase, you don't push 
   it out. If it's not ready, you send patches around, or use private git 
   trees (just as a "patch series replacement") that you don't tell the 
   public at large about.

It may also be worth noting that excessive "git rebase" will not make 
things any cleaner: if you do too many rebases, it will just mean that all 
your old pre-rebase testing is now of dubious value. So by all means 
rebase your own work, but use _some_ judgement in it.

NOTE! The combination of the above rules ("clean your own stuff" vs "don't 
clean other peoples stuff") have a secondary indirect effect. And this is 
where it starts getting subtle: since you most not rebase other peoples 
work, that means that you must never pull into a branch that isn't already 
in good shape. Because after you've done a merge, you can no longer rebase 
you commits.

Notice? Doing a "git pull" ends up being a synchronization point. But it's 
all pretty easy, if you follow these two rules about pulling:

 - Don't merge upstream code at random points. 

   You should _never_ pull my tree at random points (this was my biggest 
   issue with early git users - many developers would just pull my current 
   random tree-of-the-day into their development trees). It makes your 
   tree just a random mess of random development. Don't do it!

   And, in fact, preferably you don't pull my tree at ALL, since nothing 
   in my tree should be relevant to the development work _you_ do. 
   Sometimes you have to (in order to solve some particularly nasty 
   dependency issue), but it should be a very rare and special thing, and 
   you should think very hard about it.

   But if you want to sync up with major releases, do a

        git pull linus-repo v2.6.29

   or similar to synchronize with that kind of _non_random_ point. That 
   all makes sense. A "Merge v2.6.29 into devel branch" makes complete 
   sense as a merge message, no? That's not a problem.

   But if I see a lot of "Merge branch 'linus'" in your logs, I'm not 
   going to pull from you, because your tree has obviously had random crap 
   in it that shouldn't be there. You also lose a lot of testability, 
   since now all your tests are going to be about all my random code.

 - Don't merge _downstream_ code at random points either.

   Here the "random points" comment is a dual thing. You should not mege 
   random points as far as downstream is concerned (they should tell you 
   what to merge, and why), but also not random points as far as your tree 
   is concerned.

   Simple version: "Don't merge unrelated downstream stuff into your own 
   topic branches."

   Slightly more complex version: "Always have a _reason_ for merging 
   downstream stuff". That reason might be: "This branch is the release 
   branch, and is _not_ the 'random development' branch, and I want to 
   merge that ready feature into my release branch because it's going to 
   be part of my next release".

See? All the rules really are pretty simple. There's that somewhat subtle 
interaction between "keep your own history clean" and "never try to clean 
up _other_ proples histories", but if you follow the rules for pulling, 
you'll never have that problem.

Of course, in order for all this to work, you also have to make sure that 
the people you pull _from_ also have clean histories.

And how do you make sure of that? Complain to them if they don't. Tell 
them what they should do, and what they do wrong. Push my complaints down 
to the people you pull from. You're very much allowed to quote me on this 
and use it as an explanation of "do this, because that is what Linus 
expects from the end result".

                        Linus
|finecode

\sezione Merge driver

\iniziacode
[merge "filfre"]
       name = feel-free merge driver
       driver = gedit %O %A %B
       recursive = binary
|finecode

\capitolo Cheat Sheet

Riepilogo dei comandi \git \ e del loro agire

\input cheat-sheet

\input linkografia

\end
