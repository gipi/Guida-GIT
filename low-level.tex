\capitolo Low level

Tutti gli oggetti descritti nella struttura DAG sono compresse per una migliore
occupazione su disco tramite libreria zlib
(\url{http://www.zlib.net}) unica vera dipendenza di git; \`e possibile ottenere
un programma per la decompressione a riga di comando alla pagina
\url{http://www.zlib.net/zpipe.c}. Una volta compilato \`e possibile
decomprimere un oggetto tramite 
\iniziacode
$ zpipe -d < .git/objects/xx/
|finecode
in
maniera analoga al comando \code{git cat-file -p SHA1} bench\'e il secondo dia
un risultato formattato in maniera molto pi\`u human-readable.

\sezione Tree

Volendo essere pi\`u ``creativi'' esiste il comando \code{hash-object} che crea
un blob ed il comando \code{mktree} che crea un \code{tree} salvandoli nella
directory \code{.git/objects/} e restituendo lo \code{sha1} relativo. Infine per
creare un commit esiste il comando \code{commit-tree} che prende come primo
argomento un tree (o pi\`u d'uno) e dallo \code{stdin} il messaggio relativo;
ovviamente ci viene restituito lo \code{sha1} del commit e salvato del database
l'oggetto relativo.

L'archiviazione dei symbolic link (generati tramite \code{ln -s}) \`e effettuata
tramite il \code{tree}, un bit nei permessi imposta il contenuto al path del
file reale.
