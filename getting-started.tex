\capitolo Gitting started

In questi anni si stanno affermando un certo numero di strumenti di tipo
informatico per gestire le revisioni del codice sorgente in tutte le possibili
sfaccettature: changelog, autori, righe modificate etc... In particolare si
stanno affermando i cosidetti \evidenzia<distributed revision control>, sistemi
in cui lo sviluppo avviene in maniera distribuita senza un repository centrale e
senza la necessit\`a di collegarsi tramite rete ad esso\notapiepagina{Vedere
\url{http://people.ubuntu.com/~ianc/papers/dvcs-why-and-how.pdf}}

\sezione Glossario

Spesse volte la potenza di uno strumento si scontra con un certo numero di
concetti che \`e necessario assimilare prima di imparare ad usare
effettivamente un dato strumento e, perch\'e no, modificarlo a proprio uso e
consumo. In questa sezione definiamo alcuni termini utili per le successive
dissertazioni (alcuni termini saranno mantenuti in inglese perch\'e molto pi\`u
evocativi per me).
\medskip

\elemento SCM: Acronimo per source code management, sistema usato in ambito
ingegneristico e di svilippo software per tenere traccia dello sviluppo di
documenti di stampo elettronico, in particolare codice sorgente.

\elemento sha1: Algoritmo di hashing che genera una sequenza esadecimale di 40
cifre a partire da uno stream di dati. Pare esistano attacchi che lo rendono
insicuro ma git dovrebbe essere ragionevolmente inattaccabile.

\elemento blob: Unit\`a fondamentale nel database degli oggetti di git in quanto
memorizza il contenuto dei file. Il nome dell'oggetto effettivamente creato \`e
proprio lo sha1 del suo contenuto.

\elemento tree: Unit\`a di immagazzinamento che pu\`o contenere blob e tree
assieme ai loro metadati (quali nome e modi). Il suo nome nel filesystem \`e lo
sha1 del suo contenuto.

\elemento commit: Immagazzina un certo tree, collegando ad esso anche delle
informazioni specificanti una sua descrizione, il suo autore, la data di
creazione ed i commit da cui esso deriva. Anche in questo caso ci si riferisce
ad esso tramite lo sha1 del suo contenuto.

\elemento Repository: Collezione formata da un database di oggetti e referenze
relative a queste; praticamente implementata tramite una serie di sottodirectory
la cui radice \`e \verbatim .git/ \endverbatim.

\elemento Working tree: Rappresenta l'albero dei sorgenti fisicamente presenti
nel repository.

\elemento Index: Area in cui vengono immagazzinate le modifiche che faranno
parte del prossimo commit. Diversamente dagli altri SCM, git non memorizza i
commit direttamente dal working tree, ma proprio dall'index.

\elemento branch: \`E una referenza che indica una data linea di sviluppo in un
repository.

\elemento tag: \`E una referenza statica ad un data revisione.

\elemento master: Referenza indicante per convenzione la linea principale
(stabile) di sviluppo nel repository.

\elemento \code{HEAD}: Referenza indicante l'attuale revisione del repository. 

\elemento \code{FETCH\_HEAD}: Referenza dell'ultima operazione di fetch. 
\medskip

Alcuni di questi elementi sono presenti in molti (se non tutti) gli altri SCM;
analizziamo le caratteristiche principali di questo strumento per la revisione
del software.

\sezione Direct aciclyc graph

La struttura interna del repository di git si identifica matematicamente con il
nome di \evidenzia<grafo aciclico diretto> (nella documentazione lo potete
trovare come acronimo DAG): un grafo \`e costituito da due insiemi $(V,E)$ in
cui il primo insieme $V$ \`e l'insieme dei vertici (detti anche nodi) mentre
$E$ \`e l'insieme degli archi $(u,v)$ con $u,v\in V$.

Lo stato storico di un progetto \`e
rappresentato atomicamente con un determinato commit che descrive lo stato del
progetto attraverso il tree, il suo autore ed il commit genitore (o pi\`u di uno
eventualemente) da cui esso si origina assieme alla data in cui questo commit
\`e stato inserito nel repository.

Questa \`e la rappresentazione di un commit con i metadati relativi ad esso;
come potete notare ci sono tutte le info di cui abbiamo accennato sopra.

\includecode commit-dag

Questo punta ad un tree

\includecode tree-dag

che punta a sua volta a due oggetti (che non mostro per non inondare di inutile
merda la pagina) il cui contenuto sono il codice sorgente di questo testo (ai
suoi albori) e il suo Makefile.

Graficamente questo pu\`o essere tramite uno schema del seguente tipo
\figuratikz
\path (0:0cm) node[minimum size=1cm, fill=red!50,rounded rectangle=1cm, line width=3pt] (commit) {commit};
\path node[below right=of commit,fill=green!50,rounded rectangle] (author) {author};
\path node[below=of author,fill=green!50,rounded rectangle] (committer) {committer};
\path node[below=of committer,fill=green!50,rounded rectangle] (tree) {tree};
\path node[below right=of tree,fill=blue!50,rounded rectangle] (blob1) {blob1};
\path node[below=of blob1,fill=blue!50,rounded rectangle] (blob2) {blob2};
	\draw[->] (commit) |- (tree);
	\draw[->] (commit) |- (author);
	\draw[->] (commit) |- (committer);
	\draw[->] (tree)   |- (blob1);
  \draw[->] (tree)   |- (blob2);
\endfiguratikz[Rappresentazione del grafo di un commit]

Personalmente trovo molto utile dare una rappresentazione delle
operazioni di questo strumento: nel seguito indicheremo con
\tikz\node[draw,fill=green,shape=circle]; i vari commit che compongono il
repository e con  una referenza
interna del repository stesso; esistono vari tipi di referenze, noi indicheremo
con\tikz\node[head]{ref}; quella
di un branch oppure con \tikz\node[tag]{tag}; quella relativa ad una tag.
\figuratikz[background rectangle/.style=
	{draw=blue!50,fill=blue!20,rounded corners=1ex},
  tag/.style={anchor=tip,single arrow, scale=.5,fill=yellow!50,rotate=315,draw},
	show background rectangle]
	\tikzstyle{every node}=[draw,fill=green,shape=circle,distance=5mm];
	\path (0:0cm)    node (v0) {};
	\node (v1) [right=of v0];
	\node (v2) [right=of v1];
	\node (v3) [right=of v2];
	\node (v4) [right=of v3];
	\node (v5) [right=of v4];
	\node (w1) [below right=of v1];
	\node (w2) [right=of w1];
	\node (w3) [right=of w2];
	\draw [<-] (v0) -- (v1);
	\draw [<-] (v1) -- (v2);
	\draw [<-] (v2) -- (v3);
	\draw [<-] (v3) -- (v4);
	\draw [<-] (v4) -- (v5);
	\draw [<-] (v1) -- (w1);
	\draw [<-] (w1) -- (w2);
	\draw [<-] (w2) -- (w3);
% grazie ad "anchor=" posso impostare che la punta della freccia sia
% il punto rispetto a cui la disegna
	\node[head] (HEAD) at (v5.north) {master};
	\node[head] (branch1) at (w3.north) {topic};
  \node[tag] at (v3.north) {v1.0};
\endfiguratikz[Schema di un repository standard con due rami di sviluppo ed una
tag ``v1.0'' che identifica (per esempio) una release stabile.]

\sezione Object database o repository

Le strutture dati che effettivamente immagazinano le informazioni descritte
nella sezione precedente sono contenute nel cosidetto \evidenzia<database
bject>; questo fa parte del design interno di \git\ descritto dai suoi autori
nella seguente maniera

\citazione git is a fast, content-addressable, decentralized and
symmetrically ad hoc synchronizable, cryptographically
 verified filesystem, stored as a directed acyclic graph of
commits, trees, and blobs, with SHA-1 pointers as edges,
 backed by a POSIX-compatible filesystem, with both a
 simple storage format and a space- and seek-efficient,
          compressed, delta chained pack format.

Un punto fondamentale nel design di \git \`e che esso archivia solo i contenuti
e non per esempio i metadati dei permessi nel filesystem sottostante; questi
contenuti vengono salvati in cosidetti \evidenzia<oggetti> ed identificati
attraverso lo \evidenzia<sha1> del loro contenuto, cio\'e un checksum
rappresentato da un cifra esadecimale di 40 caratteri.
Da notare che essendo questo
algoritmo di genere crittografico abbastanza forte\notapiepagina{Esistono degli
attacchi a questo algoritmo, ma vista la caratteristica di archiviare codice
sorgente \`e plausibilmente impossibile creare un file con lo stesso
\code{sha1} avente del codice malevolo voluto.} si aggiunge un layer di
sicurezza in pi\`u all'integrità del repository.

A livello fisico gli oggetti sono salvati nella directory \code{.git/objects/}
usando delle sottodirectory con il nome composto dalle prime due cifre
esadecimali e in file con il nome dato dalle altre 38 cifre rimanenti;
per ridurre inoltre la dimensione di questo database \`e
possibile immagazzinare i vari oggetti usando dei cosidetti \evidenzia<pack
file> che ricostruiscono gli oggetti tramite le loro versioni deltificate. Essi
sono contenuti nella directory \code{.git/objects/pack/} ed elencati dal file
\code{.git/objects/info/packs}. 

Per verificare attivamente l'integrit\`a del database sono presenti strumenti
appositi quali \code{git fsck}: infatti a seconda della storia e del
comportamento dell'utente \`e possibile che all'interno del database siano
presenti oggetti non referenziati da nessuna delle referenze del repository e
che quindi eventualmente verranno eliminati con appositi comandi. \`E possibile
anche integrare/creare repository esterni a partire dagli oggetti del proprio
database, ed \`e proprio questa funzionalit\`a che permette di avere uno
sviluppo distribuito.

\sezione git

Il cuore di tutto \`e il comando \code{git} che rappresenta il cancello di
ingresso per le potenzialit\`a dello sviluppo di codice; siccome lo sviluppo \`e
molto veloce, state attenti a quale versione state utilizzando \gitversion. Allo
stato attuale il comando ha il seguente help

\includecode githelp

che d\`a una visione di insieme
delle possibilit\`a offerte. Tramite esso e le
sue opzioni \`e possibile impostare anche la directory contenente il database
del repository (usualmente \code{.git/}) oppure il workiing tree impostando
rispettivamente le opzioni \code{--git-dir=} o \code{--work-tree=} (in
alternativa esistono le variabili di ambiente \verbatim GIT_DIR |endverbatim e
\verbatim GIT_WORK_TREE|endverbatim).

Altre variabili d'ambiente utilizzate sono
\iniziaelenco
\li\verbatim GIT_INDEX_FILE |endverbatim
\li\verbatim GIT_OBJECT_DIRECTORY |endverbatim
\li\verbatim GIT_ALTERNATE_OBJECT_DIRECTORIES |endverbatim
\li\verbatim GIT_CEILING_DIRECTORIES |endverbatim
\li\verbatim GIT_AUTHOR_NAME |endverbatim
\li\verbatim GIT_AUTHOR_EMAIL |endverbatim
\li\verbatim GIT_AUTHOR_DATE |endverbatim
\li\verbatim GIT_COMMITTER_NAME |endverbatim
\li\verbatim GIT_COMMITTER_EMAIL |endverbatim
\li\verbatim GIT_COMMITTER_DATE |endverbatim
\li\verbatim GIT_DIFF_OPTS |endverbatim
\li\verbatim GIT_EXTERNAL_DIFF |endverbatim
\li\verbatim GIT_MERGE_VERBOSITY |endverbatim
\li\verbatim GIT_PAGER |endverbatim
\li\verbatim GIT_SSH |endverbatim
\li\verbatim GIT_FLUSH |endverbatim
\li\verbatim GIT_TRACE |endverbatim
\fineelenco

\sezione Revisioni

Come \`e stato detto sin dall'inizio, questo applicativo \`e un gestore di
revisioni e quindi deve essere possibile per lavorarci, disporre di una
grammatica che permetta di identificare determinate revisioni (o parti di esse)
tramite parametri temporali, di discendenza etc...
Principalmente esistono le seguenti maniere per riferirsi a delle specifiche
revisioni
\iniziaelenco
\li Il nome dell'oggetto tramite il suo \code{SHA1} che identifica il contenuto
dell'oggetto oppure una sua sottosequenza
\li Una referenza simbolica contenuta che verr\`a cercata in questo ordine
restituendo la prima trovata 
\iniziaelenco
\li\verbatim $GITDIR/<name>|endverbatim
\li\verbatim $GITDIR/refs/<name>|endverbatim
\li\verbatim $GITDIR/refs/tags/<name>|endverbatim
\li\verbatim $GITDIR/refs/heads/<name>|endverbatim
\li\verbatim $GITDIR/refs/remote/<name>|endverbatim
\li\verbatim $GITDIR/refs/remote/<name>/HEAD|endverbatim
\fineelenco
\noindent tenendo conto che esistono referenze che hanno significati particolari:
\code{HEAD} rappresenta il commit su cui l'attuale working tree \`e basato
(quindi ad ogni cambio di branch \`e lui ad essere cambiato assieme al working
tree).
\fineelenco
\`E possibile anche specificare anche una referenza derivata tramite una
notazione indiciale o temporale, la grammatica \`e la seguente:
\smallskip
\iniziaelenco
\li\verbatim <ref>@{DATE SPECIFICATION} |endverbatim
\li\verbatim <ref>@{ORDINAL SPECIFICATION} |endverbatim
\li\verbatim <rev>^ |endverbatim
\li\verbatim <rev>~N |endverbatim
\li\verbatim <ref>^{<object type>} |endverbatim
\li\verbatim <ref>^{} |endverbatim
\li\verbatim :/<some text> |endverbatim
\li\verbatim <ref>:<path> |endverbatim
\li\verbatim :{0,1,2,3}:<path> |endverbatim
\fineelenco
\smallskip
Esiste un particolare sottocomando di git per ottenere il nome della referenza
effettivamente indicata attraverso uno dei metodi appena descritti: \code{git
rev-parse}; \`e utilizzato principalmente attraverso gli script e gli altri
comandi git ma comunque rimane utile conoscere la grammatica delle referenze,
visto che sono utilizzate da tutti i sottocomandi di git.

Un'altra importantissima grammatica \`e la maniera di specificare un intervallo
di revisioni; partiamo dalle due revisioni \code{r1} e \code{r2}, se siamo
interessati ai commit raggiungibili da \code{r2} escludendo quelli
raggiungibili da \code{r1} lo si deve indicare tramite \code{r1..r2}. In maniera
similare si indica con \code{r1...r2} e restituisce l'insieme dei commit che
sono raggiungibili da uno dei due ma non entrambi (nella documentazione viene
chiamata \evidenzia<differenza simmetrica>).
Per usare parole dello stesso Torvalds
\smallskip
\noindent\line{%
source: \url{http://article.gmane.org/gmane.comp.version-control.git/80592}%
\hfil}
\smallskip
\verbatim
 - "a..b" is a plain difference. Think of it as a subtraction. For "diff", 
   it is simply the diff between a and b, and for log it is the "set 
   difference" (shown as either just "-" or as "\" in set theory math) 
   between the commits that are in b and not in a.

 - "a...b" is a more complex difference. For log, it's no longer the 
   regular set difference, but a "symmetric difference" (usually shown as 
   a greek capital "Delta" in set theory math). And for "diff", it's no 
   longer just the diff between two states, it's the diff from a third 
   state (the nearest common state) to the second state.
|endverbatim
\smallskip
Per indicare i genitori di un dato commit esistono due ulteriori abbreviazioni:
\verbatim r1^@ |endverbatim indica tutti i genitori di \code{r1}, mentre
\verbatim r1^! |endverbatim include
\code{r1} ma esclude tutti i suoi genitori (in pratica \`e una abbreviazione per
indicare un singolo commit in comandi che richiedono dei range). 

\lonelypage double-vs-triple-dot

{\bf Esempio 1:} come si comportano diversamente \code{.} e
	\code{..} rispetto a certi comandi.

\sezione Refspecs

Una refspec \`e utilizzata in comandi che si interfacciano con repository
esterni (\code{pull}, \code{push} e \code{fetch} per esempio).

Il formato di una refspec \`e un segno \code{+} opzionale, seguito da una
referenza sorgente, dal simbolo \code{:} e seguito da una referenza di
destinazione.

\sezione Installazione

Come abbiamo detto lo sviluppo di git \`e molto veloce e spesso ci si ritrova a
scoprire delle funzionalit\`a accessibili solo in fase di sviluppo
(o per chi magari
usa debian doversi ritrovare con una versione molto vecchia a causa del lungo
ciclo di rilascio di questa distribuzione). Quindi chi vuole il massimo si
consiglia di seguire queste istruzioni ed installare l'ultima versione
(preferibilmente stabile).

Prima di tutto ci sono alcune dipendenze basilari da rispettare, le librerie
\code{zlib} (fondamentali), le librerie \code{curl} (disabilitabili tramite
l'opzione \code{NO\_CURL}\notapiepagina{Consiglio di leggere \code{Makefile} per
avere qualche info in caso di installazioni particolari.} e non si avr\`a la
possibilit\`a di utilizzare \code{http://} e \code{https://} come protocolli di
trasporto.) oltre che le librerie \code{tcl-tk}; alcune dipendenze per la
documentazione: \code{asciidoc} e \code{xmlto}.

Se non lo sia ha gi\`a si scarica l'albero dei sorgenti
\iniziacode
$ wget http://www.kernel.org/pub/software/scm/git/git-<put your version here>.tar.gz
$ tar zxvf git-<put your version here>.tar.gz
|finecode
e si installa seguendo il ciclo ad ogni uscita di versione stabile (nel seguito
si indicher\`a con \code{vx.x.x.x} una versione che si intende installare)
\iniziacode
$ git fetch
$ git tag
$ git checkout -f vx.x.x.x
$ ./configure --prefix=/opt/git-vx.x.x.x
$ make
$ make install
$ make doc
$ make install-doc
$ cd /opt/
$ unlink git
$ ln -s git-vx.x.x.x git
|finecode

Da notare che la prima volta non sar\`a necessario eseguire l'unlinking della
directory \code{git} e neanche un pulling dell'albero dei sorgenti, per\`o
sar\`a necessario settare il path nella vostra shell (nel mio caso in
\code{.bashrc}) sia per gli eseguibili, per le pagine di manuale nonch\`e per le
comodissime funzionalit\`a di completamento della shell bash; il seguente pezzo
di codice permette di quanto appena detto:
\iniziacode
PATH=/opt/git/bin:$PATH
MANPATH=/opt/git/share/man/:$MANPATH

GIT_SOURCE_ROOT=/usr/src/git/

export GIT_PS1_SHOWDIRTYSTATE=1
if [ -f $GIT_SOURCE_ROOT/contrib/completion/git-completion.bash ];
then
  . $GIT_SOURCE_ROOT/contrib/completion/git-completion.bash
fi
|finecode
verificando che \code{git --version} vi restituisca la versione corretta
(inserire in quell'ordine la directory dell'installazione previene il sistema
dall'usare come eseguibile un eventuale pacchetto pi\`u vecchio).
%Purtroppo a meno di workaround non 
%conosciuti da me \code{man} non restituir\`a la pagina di
%help corretta, consiglio di usare \code{git help}.
