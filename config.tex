\capitolo Config

Il comportamento di \code{git} \`e altamente configurabile attraverso delle
impostazioni effettuate attraverso il comando \code{config} sia a livello
globale (di utente oppure di sistema usando l'opzione \code{--system})
che per ogni singolo repository (\code{.git/config}).

\def\man#1:#2\par{%
\medskip
\noindent{\tt #1}
\smallskip
\hangindent60pt\hang #2\par
}
\def\true{{\tt true}\relax}
\def\false{\code{false}}
\def\GITWORKTREE{\code{GIT\_WORK\_TREE}}
\man core.fileMode:
Se impostato a \code{false}, le differenze nell'``executable bit'' tra l'indice
e il working tree sono ignorate; utile nei filesystem come il \code{FAT}.

\man core.symlink: Se impostato a \code{false} i symbolic link sono checked out come file di testo.

\man core.autocrlf:
Se impostato a \code{true}, converte la sequenza \code{CRLF} alla fine delle linee dei
file di testo in \code{LF} quando vengono letti dal filesystem e convertiti
inversamente quando scritti. Questa variabile pu\`o anche impostata ad
\code{input}, in qual caso la conversione avviene solo in lettura, mentre la
scrittura mantiene \code{LF}. I file sono considerati testo solo in base al loro
contenuto.

\man core.safecrlf:
Se \code{true}, \code{git} controlla se la conversione come impostata da
\code{core.auto.crlf} \`e reversibile.

\man core.bare:
Se impostato a \true il repository si assume sia bare e che non abbia un working
tree associato ad esso. Di conseguenza un certo numero di comandi che si
aspettando un working tree sono disabilitati. Di default un repository che
finisce in \url{/.git} si assume che sia non bare.

\man core.worktree:
Imposta il path del working tree. Pu\`o essere reimpostato tramite la variabile
d'ambiente \GITWORKTREE o l'opzione \code{--work-tree}; se non
specificato con nessuna di queste allora si assume che il working tree sia la
directory di lavoro corrente.

\man core.excludesfile:
           In addition to .gitignore (per-directory) and .git/info/exclude,
           git looks into this file for patterns of files which are not meant
           to be tracked. See gitignore(5).

\man commit.template:
           Specify a file to use as the template for new commit messages.


