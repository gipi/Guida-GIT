\input eplain
\beginpackages
  \usepackage{url}
\endpackages
%\enablehyperlinks[pdftex]
%\hlopts{bwidth=0,colormodel=gray,color=.4}
%% TIKZ ROUTINE %%%%%%%%%%%%%%%%%%%%%%%%%%%%%%%%%%%%%%%%%%%%
\input tikz
\def\figuratikz{\noindent\midinsert\hfil\tikzpicture}
% non funzionano in locale?
\newbox\bx
\newdimen\dimbx
\newdimen\maxdimbx\maxdimbx=10cm
\newcount\figuranumero
\def\endfiguratikz[#1]{%
  \endtikzpicture%
  \global\advance\figuranumero by 1%
  \hfil%
  \smallskip%
  \noindent%
  % calcolo la dimensione del box che deve contenere
  % la didascalia e TODO: controllare che non sia
  % maggiore di un tot (tipo 10cm).
  \setbox\bx\hbox{#1}%
  \dimbx=\wd\bx%
  \ifdim\dimbx>\maxdimbx\dimbx=\maxdimbx\fi
	\line{%
    \hfil\vbox{%
      \noindent\hsize=\dimbx%
      \llap{\bf	Fig.\the\figuranumero\hfil: }\unhbox\bx\hfil%
    }\hfil}%
  \endinsert%
  \noindent%
}
\def\figuranofloattikz{\medskip\noindent\hfil\tikzpicture}
\def\finefiguranofloattikz{\endtikzpicture\hfil\medskip\noindent}
%%%% figura in mezzo al paragrafo %%%%%%
\newbox\tikzbox
\newdimen\tikzboxwt
\newdimen\tikzboxht
\newcount\lineanumero
\def\figuratikzpar{%
  \lineanumero=\the\prevgraf
  \setbox\tikzbox\hbox\bgroup\tikzpicture
}%
\def\endfiguratikzpar{%
  \endtikzpicture%
  \egroup%
  \tikzboxwt=\wd\tikzbox\advance\tikzboxwt by 1cm
  \tikzboxht=\ht\tikzbox
  \hangafter=-10
  \hangindent=\tikzboxwt
  \vadjust{%
    \smash{%
      \rlap{\hfil%
        \lower\tikzboxht\vbox{\hsize=\tikzboxwt\noindent\unhbox\tikzbox}
      \hfil}
    }
  }
}
\usetikzlibrary{shapes,positioning,folding,backgrounds}
\def\tikzcommit#1{
\bgroup
\baselineskip=5pt
Commit d47abe6767

Author: Pinco Pallo

Date:   Thu Feb 26 19:30:55 2009 +0100

\indent #1
\egroup
}
% imposto alcune tipologie di nodi
\tikzset{every node/.style={draw,fill=green,shape=circle,distance=5mm}}
\tikzset{commit/.style={shape=rectangle, rounded corners=1ex, font=\commitfn, text width=5cm}}
\tikzset{background rectangle/.style={draw=blue!50,fill=blue!20,rounded corners=1ex}}
\tikzset{tag/.style={anchor=tip,single arrow, scale=.5,fill=yellow!50,rotate=315,draw}}
\tikzset{head/.style={anchor=tip,single arrow, scale=.5,fill=red!50,rotate=315,draw}}
%%%%%%%%%%%%%%%%%%%%%%%%%%%%%%%%%%%%%%%%%%%%%%%%%%%%%%%%%%%%
\font\capitolofont=cmssdc10 at 30pt
\font\sezionefont=cmssdc10 at 15pt
%	from tex-by-topic pg211
\def\goodbreak{\par\penalty-500 } % encourage page break
\def\capitolo#1\par{%
	\vfill\supereject
	\noindent\line{\capitolofont #1\hfil}%
	\vskip 2cm
}%
\def\sezione#1\par{%
	\bigskip
	\goodbreak
	\noindent\line{\sezionefont #1\hfil}%
	\smallskip\noindent
}%
\newcount\numeronota
\let\notapiepagina=\numberedfootnote
\def\elemento#1: #2\par{% from 'A plain Tex primer'
	\smallskip
	\bgroup
	\parindent=4cm
	\narrower
	\noindent\llap{\hbox to 4cm{\hfil\bf #1\hfil}}#2\par%
	\egroup
	\smallskip
}%
\def\code #1{{\tt #1}}
\def\iniziacode{\smallskip\noindent\line{\hrulefill}\smallskip\verbatim}
\edef\finecode{%\endverbatim
  \smallskip\noindent\line{\hrulefill}\goodbreak\smallskip\noindent}
\def\figura#1\par{%
	\midinsert%
		% da mettere a posto
		\medskip
		\line{\hfil\epsfbox{#1}\hfil}
	\endinsert\noindent
}%
\def\evidenzia<#1>{{\bf #1}}
\def\link#1: #2\par{%
	\goodbreak
	\noindent\url{#1}
	\smallskip
	\hfil\vbox{\advance\hsize by-1cm\noindent #2}
	\medskip
}
%%%%%%%%%%%%%%%%%%%%%%%%%%%%%%%%%%%%%%%%%%%%%%%%%%%%%%%%%%%
\def\licenza{%
\vbox{Permission is granted to copy, distribute and/or modify the documentation
under the terms of the gnu Free Documentation License, Version 1.2
or any later version published by the Free Software Foundation;
with no Invariant Sections, no Front-Cover Texts, and no Back-Cover Texts.
A copy of the license is included in the section entitled gnu Free
Documentation License. Permission is granted to copy, distribute and/or modify
the code of the package under the terms of the gnu Public License, Version 2 or
any later version published by the Free Software Foundation. A copy of the
license is included in the section entitled gnu Public License.}
}
\def\gitversion{\input versione }
\def\githelp{\smallskip\input githelp \smallskip}
\def\GITDIR#1{\code{$GIT_DIR/#1}}
