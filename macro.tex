\input eplain
\beginpackages
  \usepackage{url}
\endpackages
%%%%%%%% SALVIAMO QUALCHE PARAMETRO COME È DI DEFAULT %%%%%%%%%%
\newdimen\defaultparskip\defaultparskip=\parskip
\newdimen\defaultparindent\defaultparindent=\parindent
%%%%%%%%%%%%%%%%%%%%%%%%%%%%%%%%%%%%%%%%%%%%%%%%%%%%%%%%%%%%%%%%
%\enablehyperlinks[pdftex]
%\hlopts{bwidth=0,colormodel=gray,color=.4}
\font\defaultfn=cmr10
\font\capitolofont=cmssdc10 at 30pt
\font\sezionefont=cmssdc10 at 15pt
\font\footnotefont=cmr10 at 7pt
%	from tex-by-topic pg211
\def\goodbreak{\par\penalty-500 } % encourage page break
\def\capitolo#1\par{%
	\vfill\supereject
	\noindent\line{\capitolofont #1\hfil}%
	\vskip 2cm
}%
\def\sezione#1\par{%
	\bigskip
	\goodbreak
	\noindent\line{\sezionefont #1\hfil}%
	\smallskip\noindent
}%
\newcount\numeronota
% eplain reference pg30
\let\notapiepagina=\numberedfootnote
\everyfootnote={\parindent=\defaultparindent}
\def\elemento#1: #2\par{% from 'A plain Tex primer'
	\smallskip
	\bgroup
	\parindent=4cm
	\narrower
	\noindent\llap{\hbox to 4cm{\hfil\bf #1\hfil}}#2\par%
	\egroup
	\smallskip
}%
%%%%%%%%%%%%%%% VERBATIM STUFFS %%%%%%%%%%%%%%%%%%
\def\code #1{{\tt #1}}
\def\cc #1 {{\tt #1}}

\def\iniziacode{%
  \bgroup
  \parskip=\defaultparskip
  \smallskip\noindent
  \smallskip\verbatim}

\edef\finecode{
  \endverbatim
  \smallskip
  \egroup
  \goodbreak\noindent}

\def\includecode#1\par{%
\medskip
\listing{#1}
\medskip
\noindent
}
%%%%%%%%%%%%%% END VERBATIM STUFFS %%%%%%%%%%%%%%%%%%

\def\figura#1\par{%
	\midinsert%
		% da mettere a posto
		\medskip
		\line{\hfil\epsfbox{#1}\hfil}
	\endinsert\noindent
}%
% per le cose importanti
\def\evidenzia<#1>{{\bf #1}}
% per i link?
\def\link#1: #2\par{%
	\goodbreak
	\noindent\url{#1}
	\smallskip
	\hfil\vbox{\advance\hsize by-1cm\noindent #2}
	\medskip
}
% per le citazioni
\def\citazione#1\par{
	\bgroup
	\bigskip
	\parindent=2cm\narrower\sl
	\noindent #1
	\bigskip
	\egroup
}
%%%%%%%%%%%%%% CODE SAMPLES %%%%%%%%%%%%%%%%%%%%%%%%%%%%%%%
\def\lonelypage#1\par#2\par{%
	\pageinsert
	\line{\hrulefill}%
	\input #1 
	\medskip
	\centerline{#2}
	\line{\hrulefill}
	\vfill
	\endinsert
}
%%%%%%%%%%%%%%%%%%%%%%%%%%%%%%%%%%%%%%%%%%%%%%%%%%%%%%%%%%%
\def\iniziaelenco{
  \medskip
  \bgroup
  \multiply \parindent by 3
  \unorderedlist
}
\def\fineelenco{
  \endunorderedlist
  \egroup
  \medskip\noindent
}
\def\licenza{%
\vbox{Permission is granted to copy, distribute and/or modify the documentation
under the terms of the gnu Free Documentation License, Version 1.2
or any later version published by the Free Software Foundation;
with no Invariant Sections, no Front-Cover Texts, and no Back-Cover Texts.
A copy of the license is included in the section entitled gnu Free
Documentation License. Permission is granted to copy, distribute and/or modify
the code of the package under the terms of the gnu Public License, Version 2 or
any later version published by the Free Software Foundation. A copy of the
license is included in the section entitled gnu Public License.}
}
\def\gitversion{\input versione }
\def\githelp{\smallskip\input githelp \smallskip}
\def\GITDIR#1{\code{$GIT_DIR/#1}}
\def\git{\code{git}}
\def\HEAD{\code{HEAD}}
\def\URL{\code{URL}}
% TODO: \path
\def\newpage{\vfil\eject}

% TODO: creare ambiente ala minipage in cui si crea un qualcosa
% staccato dal flow normale del documento che vada in una pagina a parte.
\def\minipage#1\par{%
  \pageinsert
  \vbox{#1}
  \vfill
  \centerline{{\bf Fig:} figura interessante}
  \endinsert
}%
