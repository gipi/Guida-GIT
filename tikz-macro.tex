%% TIKZ ROUTINE %%%%%%%%%%%%%%%%%%%%%%%%%%%%%%%%%%%%%%%%%%%%
\input tikz
\usetikzlibrary{calc}
\def\figuratikz{\noindent\midinsert\hfil\tikzpicture}
% non funzionano in locale?
\newbox\bx
\newdimen\dimbx
\newdimen\maxdimbx\maxdimbx=10cm
\newcount\figuranumero
\def\endfiguratikz[#1]{%
  \endtikzpicture%
  \global\advance\figuranumero by 1%
  \hfil%
  \smallskip%
  \noindent%
  % calcolo la dimensione del box che deve contenere
  % la didascalia e TODO: controllare che non sia
  % maggiore di un tot (tipo 10cm).
  \setbox\bx\hbox{#1}%
  \dimbx=\wd\bx%
  \ifdim\dimbx>\maxdimbx\dimbx=\maxdimbx\fi
	\line{%
    \hfil\vbox{%
      \noindent\hsize=\dimbx%
      \llap{\bf	Fig.\the\figuranumero\hfil: }\unhbox\bx\hfil%
    }\hfil}%
  \endinsert%
  \noindent%
}
\def\figuranofloattikz{\medskip\noindent\hfil\tikzpicture}
\def\finefiguranofloattikz{\endtikzpicture\hfil\medskip\noindent}
%%%% figura in mezzo al paragrafo %%%%%%
\newbox\tikzbox
\newdimen\tikzboxwt
\newdimen\tikzboxht
\newcount\lineanumero
\def\figuratikzpar{%
  \lineanumero=\the\prevgraf
  \setbox\tikzbox\hbox\bgroup\tikzpicture
}%
\def\endfiguratikzpar{%
  \endtikzpicture%
  \egroup%
  \tikzboxwt=\wd\tikzbox\advance\tikzboxwt by 1cm
  \tikzboxht=\ht\tikzbox
  \hangafter=-10
  \hangindent=\tikzboxwt
  \vadjust{%
    \smash{%
      \rlap{\hfil%
        \lower\tikzboxht\vbox{\hsize=\tikzboxwt\noindent\unhbox\tikzbox}
      \hfil}
    }
  }
}
\usetikzlibrary{shapes,positioning,folding,backgrounds}
\def\tikzcommit#1{
\bgroup
\baselineskip=5pt
Commit d47abe6767

Author: Pinco Pallo

Date:   Thu Feb 26 19:30:55 2009 +0100

\indent #1
\egroup
}
% imposto alcune tipologie di nodi
%
% tutti i nodi relativi a commit/tree/blob devono avere
% \font bold
% line width 1pt
\font\yeahdummuy=cmssbx10 at 7pt
%\tikzset{every node/.style={draw,fill=green,shape=circle,distance=5mm}}
\tikzset{commit/.style={shape=rounded rectangle, draw, fill=green, font=\yeahdummuy}}
\tikzset{logcommit/.style={shape=rectangle, rounded corners=1ex, font=\commitfn, text width=5cm}}
\tikzset{background rectangle/.style={draw=blue!50,fill=blue!20,rounded corners=1ex}}
\tikzset{tag/.style={anchor=tip, single arrow, scale=.5,fill=yellow!50,rotate=315,draw}}
\tikzset{head/.style={anchor=tip, single arrow,
scale=.7,fill=red!50,shape border rotate=270,draw, outer sep=3pt, font=\yeahdummuy}}
\tikzset{HEAD/.style={head,fill=white}}
\tikzset{index/.style={rounded rectangle, fill=blue!20!red, font=\yeahdummuy}}
\tikzset{wt/.style={rounded rectangle, fill=blue!20!green, font=\yeahdummuy}}
% qualche variabile utile %%%%%%%%%%%%%%%%%%%%%%%%%%%%%%%
% questa definisce se scrivere o meno il nodo dentro il commit
\newif\ifNodeName
\NodeNametrue
\newtoks\nodeName
% #testo #nodo
\def\rootCommit #1;{
  \nodeName{\ifNodeName#1\else{}\fi}
  \path (0:0cm) node[commit] (#1) {\the\nodeName};
}
\def\metaCommit #1 <- #2 [#3];{
  \nodeName{\ifNodeName#2\else{}\fi}
  \node[commit] (#2) [#3=of #1] {\the\nodeName};
  \draw [<-] (#1) -- (#2);
}
% controintuitivo: #vecchio nodo <- #nuovo nodo
\def\commit #1 <- #2;{
  \metaCommit #1 <- #2 [right];
}
% controintuitivo: #vecchio nodo <- #nuovo nodo
\def\upCommit #1 <- #2;{
  \metaCommit #1 <- #2 [above right];
}
% controintuitivo: #vecchio nodo <- #nuovo nodo
\def\downCommit #1 <- #2;{
  \metaCommit #1 <- #2 [below right];
}
\def\mergeCommit #1 - #2 <- #3;{
  \node[commit] (#3) [right=of #1] {#3};
  \draw [<-] (#1) -- (#3);
  \draw [<-] (#2) -- (#3);
}
% crea un commit sotto il #1 e si suppone che #2 sia sotto-sotto
\def\mergeCommitInTheMiddle #1 - #2 <- #3;{
  \node[commit] (#3) [below right=of #1] {#3};
  \draw [<-] (#1) -- (#3);
  \draw [<-] (#2) -- (#3);
}
% #testo #nodo a cui puntare
\def\branchref #1 -> #2;{
  \node[head] (#1) at (#2.north) {#1};
}
\def\HEADref -> #1;{
  \node[HEAD] (HEAD) at (#1.north) {HEAD};
}
%%%%%%%%%%%%%%%%%%%%%%%%%%%%%%%%%%%%%%%%%%%%%%%%%%%%%%%%%%%%
